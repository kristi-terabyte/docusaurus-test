% !TEX program = pdflatex
% Курсова робота: Інформаційна система ательє
% Структура оформлена відповідно до ДСТУ (ЗМІСТ, нумерація, посилання)
\documentclass[14pt,a4paper]{extarticle}
\usepackage[utf8]{inputenc}
\usepackage[T2A]{fontenc}
\usepackage[ukrainian]{babel}
\usepackage{geometry}
\geometry{left=30mm,right=15mm,top=20mm,bottom=20mm}
\usepackage{setspace}
\onehalfspacing
\usepackage{indentfirst}
\usepackage{hyperref}
\hypersetup{colorlinks=true,linkcolor=black,urlcolor=blue,citecolor=black}
\usepackage{graphicx}
\usepackage{float}
\usepackage{caption}
\usepackage{subcaption}
\usepackage{longtable}
\usepackage{booktabs}
\usepackage{enumitem}
\usepackage{amsmath,amssymb}
\usepackage{tikz}
\usetikzlibrary{arrows.meta,positioning,shapes.geometric,shapes.misc,fit,backgrounds}

% Зручні стилі для діаграм
\tikzset{
  actor/.style={draw,rounded corners,minimum height=7mm,minimum width=16mm,fill=gray!10},
  usecase/.style={draw,ellipse,minimum height=8mm,minimum width=24mm,fill=blue!5},
  comp/.style={draw,rectangle,rounded corners,fill=green!5,minimum width=28mm,minimum height=10mm},
  db/.style={draw, cylinder, shape border rotate=90, aspect=0.2, minimum height=13mm, minimum width=10mm, fill=orange!10},
  entity/.style={draw,rectangle,fill=yellow!10,rounded corners},
  class/.style={draw,rectangle,fill=purple!5,rounded corners,align=left},
  activity/.style={draw,rectangle,rounded corners,fill=cyan!10},
  startstop/.style={draw,ellipse,fill=black!5},
  edge/.style={-Latex,thick},
}

\begin{document}

% Титульна сторінка
\begin{titlepage}
  \centering
  \large
  Міністерство освіти і науки України\\
  [1ex]
  Назва університету\\
  Кафедра інформаційних систем\\[12ex]
  \textbf{КУРСОВА РОБОТА}\\[1ex]
  з дисципліни: «Проєктування інформаційних систем»\\[1ex]
  на тему: \textbf{«Інформаційна система ательє»}\\[12ex]
  Виконав(ла): Студент(ка) групи ХХ-ХХ\\
  Керівник: ПІБ, науковий ступінь\\[10ex]
  Місто — 2025
\end{titlepage}

% ЗМІСТ (відповідно до ДСТУ)
\tableofcontents
\newpage

\section*{Перелік умовних скорочень}
\addcontentsline{toc}{section}{Перелік умовних скорочень}
ІС — інформаційна система; БД — база даних; UML — Unified Modeling Language; ER — Entity–Relationship.

\section{Вступ}
Мета роботи — спроєктувати та описати ІС для ательє: облік клієнтів, замовлень, послуг, матеріалів, розрахунки, звітність. Завдання: аналіз предметної області, визначення вимог, проєктування структури даних, розробка діаграм (прецедентів, класів, ER, активностей, архітектури), опис інтерфейсів та алгоритмів.

\section{Огляд предметної області і вимоги}
\subsection{Актори}
Клієнт; Адміністратор (приймання замовлень); Майстер; Менеджер (матеріали, постачання); Бухгалтер/Система оплат.
\subsection{Функціональні вимоги}
\begin{enumerate}[label=F\arabic*.]
  \item Реєстрація та ведення картки клієнта.
  \item Приймання замовлення: вибір послуг, зняття мірок, терміни, передплата.
  \item Планування і виконання робіт майстрами; відстеження статусів.
  \item Облік матеріалів і витрат.
  \item Розрахунки, чеки, акти; закриття замовлення і видача виробу.
  \item Звіти: завантаження майстрів, прибутковість, матеріаломісткість.
\end{enumerate}
\subsection{Нефункціональні вимоги}
Доступність, цілісність даних, аудит змін, рольовий доступ, резервне копіювання, масштабованість.

\section{Діаграма прецедентів (Use Case)}
\begin{figure}[H]
  \centering
  \begin{tikzpicture}[node distance=12mm]
    % Актори
    \node[actor] (client) {Клієнт};
    \node[actor, below=of client] (admin) {Адміністратор};
    \node[actor, below=of admin] (master) {Майстер};
    \node[actor, below=of master] (manager) {Менеджер};
    \node[actor, below=of manager] (accountant) {Бухгалтер};

    % Межі системи
    \node[draw, dashed, fit={(6,-4) (15,4)}, inner sep=6mm, label=above:{ІС Ательє}] (system) {};

    % Прецеденти
    \node[usecase] (reg) at (9,3) {Вести клієнтів};
    \node[usecase] (order) at (11,1.5) {Оформити замовлення};
    \node[usecase] (measure) at (13,3) {Зняти мірки};
    \node[usecase] (schedule) at (9,0) {Планувати роботи};
    \node[usecase] (execute) at (13,0) {Виконати роботи};
    \node[usecase] (materials) at (9,-1.8) {Облік матеріалів};
    \node[usecase] (payment) at (13,-1.8) {Оплати і чеки};
    \node[usecase] (close) at (11,-3.2) {Видати виріб і закрити};

    % Зв'язки акторів
    \draw[edge] (admin.east) -- (reg.west);
    \draw[edge] (client.east) -- (order.west);
    \draw[edge] (admin.east) -- (order.west);
    \draw[edge] (admin.east) -- (measure.west);
    \draw[edge] (manager.east) -- (materials.west);
    \draw[edge] (master.east) -- (execute.west);
    \draw[edge] (accountant.east) -- (payment.west);
    \draw[edge] (admin.east) -- (close.west);

    % include/extend (спрощено підписом)
    \draw[edge] (order) -- node[above]{include} (measure);
    \draw[edge] (close) -- node[right]{extend} (payment);
  \end{tikzpicture}
  \caption{Діаграма прецедентів ІС ательє}
\end{figure}

\section{Архітектура та структура рішення}
\begin{figure}[H]
  \centering
  \begin{tikzpicture}[node distance=10mm]
    \node[comp] (ui) {UI: Web/Mobile};
    \node[comp, below=of ui] (api) {API: REST/GraphQL};
    \node[comp, below=of api] (service) {Сервіси: Замовлення, Клієнти, Склад, Платежі};
    \node[db, right=25mm of service] (db) {БД};
    \node[comp, left=25mm of service] (auth) {Аутентифікація/Авторизація};

    \draw[edge] (ui) -- (api);
    \draw[edge] (api) -- (service);
    \draw[edge] (service) -- (db);
    \draw[edge] (api) -- (auth);
  \end{tikzpicture}
  \caption{Високорівнева архітектура}
\end{figure}
Коротко: клієнтські застосунки взаємодіють з API, доменні сервіси інкапсулюють логіку, дані зберігаються у БД; окремий сервіс автентифікації.

\section{ER-діаграма бази даних}
\begin{figure}[H]
  \centering
  \begin{tikzpicture}[node distance=12mm]
    \node[entity] (client) {КЛІЄНТ\\id, ПІБ, телефон, email, адреса};
    \node[entity, right=30mm of client] (order) {ЗАМОВЛЕННЯ\\id, client\_id, дата, статус, передплата, сума};
    \node[entity, below=of order] (orderitem) {ПОЗИЦІЯ\\id, order\_id, послуга\_id, ціна, кількість};
    \node[entity, below=of client] (measure) {МІРКИ\\id, client\_id, параметри, дата};
    \node[entity, right=30mm of order] (service) {ПОСЛУГА\\id, назва, тариф};
    \node[entity, below=of service] (material) {МАТЕРІАЛ\\id, назва, од., ціна, залишок};
    \node[entity, below=of orderitem] (usage) {ВИТРАТА\\id, order\_id, material\_id, кількість};
    \node[entity, below=of usage] (payment) {ПЛАТІЖ\\id, order\_id, дата, сума, тип};

    \draw[edge] (client) -- node[above]{1..*} (order);
    \draw[edge] (order) -- node[right]{1..*} (orderitem);
    \draw[edge] (client) -- node[left]{1..*} (measure);
    \draw[edge] (orderitem) -- node[above]{*..1} (service);
    \draw[edge] (usage) -- node[above]{*..1} (material);
    \draw[edge] (order) -- node[right]{1..*} (usage);
    \draw[edge] (order) -- node[right]{1..*} (payment);
  \end{tikzpicture}
  \caption{ER-модель даних ательє}
\end{figure}

\section{Діаграма класів}
\begin{figure}[H]
  \centering
  \begin{tikzpicture}[node distance=10mm]
    \node[class] (Client) {\textbf{Client}\\+id: int\\+fullName: string\\+phone: string\\+email: string\\+address: string\\+getActiveOrders()};
    \node[class, right=25mm of Client] (Order) {\textbf{Order}\\+id: int\\+clientId: int\\+date: date\\+status: OrderStatus\\+prepaid: money\\+total: money\\+calculateTotal()};
    \node[class, below=of Order] (OrderItem) {\textbf{OrderItem}\\+id: int\\+orderId: int\\+serviceId: int\\+price: money\\+qty: number};
    \node[class, above=of Order] (Measurement) {\textbf{Measurement}\\+id: int\\+clientId: int\\+params: json\\+date: date};
    \node[class, right=30mm of Order] (Service) {\textbf{Service}\\+id: int\\+name: string\\+rate: money};
    \node[class, right=30mm of OrderItem] (Material) {\textbf{Material}\\+id: int\\+name: string\\+uom: string\\+price: money\\+stock: number\\+reserve(qty)};
    \node[class, below=of OrderItem] (Payment) {\textbf{Payment}\\+id: int\\+orderId: int\\+date: date\\+amount: money\\+type: PaymentType};

    \draw[edge] (Client) -- node[above]{1..*} (Order);
    \draw[edge] (Order) -- node[right]{1..*} (OrderItem);
    \draw[edge] (Client) -- node[left]{1..*} (Measurement);
    \draw[edge] (OrderItem) -- node[above]{*..1} (Service);
    \draw[edge] (Order) -- node[right]{1..*} (Payment);
  \end{tikzpicture}
  \caption{Діаграма класів}
\end{figure}

\section{Діаграма активностей (workflow) замовлення}
\begin{figure}[H]
  \centering
  \begin{tikzpicture}[node distance=8mm]
    \node[startstop] (start) {Старт};
    \node[activity, below=of start] (create) {Приймання замовлення};
    \node[activity, below=of create] (measure) {Зняття мірок};
    \node[activity, below=of measure] (plan) {Планування та призначення майстра};
    \node[activity, below=of plan] (exec) {Виконання робіт};
    \node[activity, below=of exec] (qc) {Примірка/контроль якості};
    \node[activity, below=of qc] (payment) {Оплата, закриття};
    \node[startstop, below=of payment] (end) {Кінець};

    \draw[edge] (start) -- (create) -- (measure) -- (plan) -- (exec) -- (qc) -- (payment) -- (end);
  \end{tikzpicture}
  \caption{Activity-діаграма життєвого циклу замовлення}
\end{figure}

\section{Проєктування інтерфейсу користувача}
Сторінки: реєстр клієнтів, картка клієнта з мірками; реєстр замовлень; картка замовлення (послуги, матеріали, платежі, статуси); звіти.

\section{Економічне обґрунтування та план впровадження}
Кошторис на розробку, витрати на інфраструктуру, очікуваний ефект (скорочення часу обслуговування, прозорість обліку), ризики і заходи зниження.

\section{Висновки}
Розроблено структурну модель ІС ательє, спроєктовано БД та основні UML-діаграми; визначено ключові функції та вимоги.

\section*{Список використаних джерел}
\addcontentsline{toc}{section}{Список використаних джерел}
1. Fowler M. UML Distilled.\\
2. Ambler S. Agile Modeling.\\
3. ISO/IEC/IEEE 42010:2011.\\
4. ДСТУ 3008:2015 Документація. Звіти у сфері науки і техніки.

\end{document}
