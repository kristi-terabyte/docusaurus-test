\documentclass[14pt,a4paper]{extarticle}
\usepackage[utf8]{inputenc}
\usepackage[ukrainian]{babel}
\usepackage{geometry}
\geometry{left=3cm,right=1cm,top=2cm,bottom=2cm}
\usepackage{indentfirst}
\usepackage{setspace}
\onehalfspacing
\usepackage{tikz}
\usepackage{pgf-umlcd}
\usepackage{pgf-umlsd}
\usetikzlibrary{shapes,arrows,positioning,calc,fit}
\usepackage{graphicx}
\usepackage{listings}
\usepackage{xcolor}
\usepackage{hyperref}
\usepackage{tocloft}
\usepackage{titlesec}

% Налаштування відступів для розділів
\titleformat{\section}
  {\normalfont\fontsize{14}{16}\bfseries}{\thesection}{1em}{}
\titleformat{\subsection}
  {\normalfont\fontsize{14}{16}\bfseries}{\thesubsection}{1em}{}

% Налаштування змісту
\renewcommand{\cftsecleader}{\cftdotfill{\cftdotsep}}

\begin{document}

% Титульна сторінка
\begin{titlepage}
\centering
\textbf{МІНІСТЕРСТВО ОСВІТИ І НАУКИ УКРАЇНИ}\\
\textbf{[НАЗВА УНІВЕРСИТЕТУ]}\\[0.5cm]
\textbf{Факультет [назва факультету]}\\
\textbf{Кафедра [назва кафедри]}\\[3cm]

{\Large\textbf{КУРСОВА РОБОТА}}\\[0.5cm]
з дисципліни «[Назва дисципліни]»\\[0.5cm]
на тему:\\[0.5cm]
{\Large\textbf{«ІНФОРМАЦІЙНА СИСТЕМА УПРАВЛІННЯ\\АТЕЛЬЄ»}}\\[3cm]

\begin{flushright}
Виконав:\\
студент групи [номер групи]\\
[ПІБ студента]\\[1cm]
Перевірив:\\
[ПІБ викладача]\\
[посада, науковий ступінь]
\end{flushright}

\vfill

Місто, 2025
\end{titlepage}

\newpage
\tableofcontents
\newpage

\section{ВСТУП}

\subsection{Актуальність теми}

Сучасний розвиток інформаційних технологій створює нові можливості для автоматизації бізнес-процесів у різних галузях економіки. Ательє як підприємство, що надає послуги з пошиття та ремонту одягу, потребує ефективної системи управління для підвищення якості обслуговування клієнтів та оптимізації виробничих процесів.

Актуальність розробки інформаційної системи для ательє обумовлена наступними факторами:
\begin{itemize}
    \item необхідність автоматизації обліку замовлень та клієнтів;
    \item потреба у контролі виконання замовлень та завантаженості працівників;
    \item важливість ведення складського обліку матеріалів;
    \item необхідність формування звітності для аналізу діяльності підприємства;
    \item підвищення конкурентоспроможності за рахунок покращення якості обслуговування.
\end{itemize}

\subsection{Мета та завдання курсової роботи}

\textbf{Мета роботи:} розробка інформаційної системи управління ательє для автоматизації основних бізнес-процесів підприємства.

\textbf{Завдання роботи:}
\begin{enumerate}
    \item Проаналізувати предметну область та виявити основні бізнес-процеси ательє;
    \item Визначити функціональні вимоги до інформаційної системи;
    \item Розробити діаграми прецедентів для визначення взаємодії користувачів з системою;
    \item Створити діаграму класів для опису структури системи;
    \item Розробити модель бази даних;
    \item Описати архітектуру системи;
    \item Розробити діаграми діяльності для ключових бізнес-процесів.
\end{enumerate}

\subsection{Об'єкт та предмет дослідження}

\textbf{Об'єкт дослідження:} бізнес-процеси ательє з пошиття та ремонту одягу.

\textbf{Предмет дослідження:} методи та засоби автоматизації управління діяльністю ательє.

\newpage
\section{АНАЛІЗ ПРЕДМЕТНОЇ ОБЛАСТІ}

\subsection{Опис предметної області}

Ательє --- це підприємство, що надає послуги з пошиття одягу на замовлення, ремонту та підгонки одягу за розміром клієнта. Основними процесами ательє є:

\begin{itemize}
    \item \textbf{Прийом замовлень} --- консультація клієнта, зняття мірок, вибір тканини та фасону, визначення вартості та термінів виконання;
    \item \textbf{Виробництво} --- розкрій тканини, пошиття виробу, примірка та коригування;
    \item \textbf{Видача замовлень} --- перевірка якості, розрахунок з клієнтом, видача готового виробу;
    \item \textbf{Складський облік} --- облік тканин, фурнітури та інших матеріалів;
    \item \textbf{Управління персоналом} --- розподіл замовлень між кравцями, облік робочого часу.
\end{itemize}

\subsection{Основні сутності предметної області}

\begin{enumerate}
    \item \textbf{Клієнт} --- особа, яка замовляє послуги ательє;
    \item \textbf{Замовлення} --- конкретне замовлення клієнта на виготовлення або ремонт виробу;
    \item \textbf{Працівник} --- співробітник ательє (кравець, закрійник, адміністратор);
    \item \textbf{Послуга} --- вид роботи (пошиття сукні, ремонт брюк, підгонка піджака тощо);
    \item \textbf{Матеріал} --- тканина, фурнітура та інші матеріали;
    \item \textbf{Мірки} --- індивідуальні параметри клієнта.
\end{enumerate}

\subsection{Функціональні вимоги до системи}

Інформаційна система управління ательє повинна забезпечувати:

\begin{enumerate}
    \item \textbf{Управління клієнтами:}
    \begin{itemize}
        \item реєстрація нових клієнтів;
        \item зберігання контактної інформації;
        \item зберігання мірок клієнтів;
        \item історія замовлень клієнта.
    \end{itemize}
    
    \item \textbf{Управління замовленнями:}
    \begin{itemize}
        \item створення нових замовлень;
        \item відстеження статусу виконання;
        \item призначення виконавців;
        \item розрахунок вартості;
        \item контроль термінів виконання.
    \end{itemize}
    
    \item \textbf{Управління матеріалами:}
    \begin{itemize}
        \item облік наявності матеріалів на складі;
        \item облік витрачання матеріалів на замовлення;
        \item контроль залишків та формування замовлень постачальникам.
    \end{itemize}
    
    \item \textbf{Управління персоналом:}
    \begin{itemize}
        \item облік працівників;
        \item розподіл замовлень між виконавцями;
        \item контроль завантаженості.
    \end{itemize}
    
    \item \textbf{Звітність:}
    \begin{itemize}
        \item звіт по виконаних замовленнях;
        \item фінансовий звіт;
        \item звіт по завантаженості працівників;
        \item звіт по залишках матеріалів.
    \end{itemize}
\end{enumerate}

\newpage
\section{ПРОЕКТУВАННЯ ІНФОРМАЦІЙНОЇ СИСТЕМИ}

\subsection{Діаграма прецедентів}

Діаграма прецедентів (use case diagram) відображає взаємодію користувачів з системою та основні функції, які вона надає.

\begin{figure}[h!]
\centering
\begin{tikzpicture}[scale=0.9, every node/.style={scale=0.85}]
    % Стиль для акторів
    \tikzstyle{actor}=[stick figure, minimum size=1.2cm, thick]
    
    % Актори зліва
    \node at (-4,6) {\textbf{Клієнт}};
    \draw[thick] (-4,5.5) circle (0.3cm);
    \draw[thick] (-4,5.2) -- (-4,4.2);
    \draw[thick] (-4.5,4.7) -- (-4,5) -- (-3.5,4.7);
    \draw[thick] (-4.5,3.7) -- (-4,4.2) -- (-3.5,3.7);
    
    \node at (-4,2) {\textbf{Адміністратор}};
    \draw[thick] (-4,1.5) circle (0.3cm);
    \draw[thick] (-4,1.2) -- (-4,0.2);
    \draw[thick] (-4.5,0.7) -- (-4,1) -- (-3.5,0.7);
    \draw[thick] (-4.5,-0.3) -- (-4,0.2) -- (-3.5,-0.3);
    
    \node at (-4,-3) {\textbf{Кравець}};
    \draw[thick] (-4,-3.5) circle (0.3cm);
    \draw[thick] (-4,-3.8) -- (-4,-4.8);
    \draw[thick] (-4.5,-4.3) -- (-4,-4) -- (-3.5,-4.3);
    \draw[thick] (-4.5,-5.3) -- (-4,-4.8) -- (-3.5,-5.3);
    
    % Межа системи
    \draw[thick] (-0.5,-6.5) rectangle (11,7);
    \node at (5.2,6.6) {\textbf{Система управління ательє}};
    
    % Прецеденти для клієнта
    \node[ellipse, draw, thick, text width=2.2cm, align=center, minimum height=1.2cm] (uc1) at (2,5.5) {\small Переглянути\\каталог послуг};
    \node[ellipse, draw, thick, text width=2.2cm, align=center, minimum height=1.2cm] (uc2) at (6,5.5) {\small Оформити\\замовлення};
    \node[ellipse, draw, thick, text width=2.2cm, align=center, minimum height=1.2cm] (uc3) at (9,5.5) {\small Переглянути\\статус};
    
    % Прецеденти для адміністратора
    \node[ellipse, draw, thick, text width=2.2cm, align=center, minimum height=1.2cm] (uc4) at (1,2.5) {\small Керування\\клієнтами};
    \node[ellipse, draw, thick, text width=2.2cm, align=center, minimum height=1.2cm] (uc5) at (4,2.5) {\small Керування\\замовленнями};
    \node[ellipse, draw, thick, text width=2.2cm, align=center, minimum height=1.2cm] (uc6) at (7,2.5) {\small Керування\\послугами};
    \node[ellipse, draw, thick, text width=2.2cm, align=center, minimum height=1.2cm] (uc7) at (1,-0.5) {\small Керування\\матеріалами};
    \node[ellipse, draw, thick, text width=2.2cm, align=center, minimum height=1.2cm] (uc8) at (4,-0.5) {\small Керування\\персоналом};
    \node[ellipse, draw, thick, text width=2.2cm, align=center, minimum height=1.2cm] (uc9) at (7,-0.5) {\small Формування\\звітів};
    
    % Прецеденти для кравця
    \node[ellipse, draw, thick, text width=2.2cm, align=center, minimum height=1.2cm] (uc10) at (2.5,-4) {\small Переглянути\\свої замовлення};
    \node[ellipse, draw, thick, text width=2.2cm, align=center, minimum height=1.2cm] (uc11) at (7,-4) {\small Оновити\\статус};
    
    % Зв'язки
    \draw[thick] (-3.5,5) -- (uc1);
    \draw[thick] (-3.5,5) -- (uc2);
    \draw[thick] (-3.5,5) -- (uc3);
    
    \draw[thick] (-3.5,0.5) -- (uc4);
    \draw[thick] (-3.5,0.5) -- (uc5);
    \draw[thick] (-3.5,0.5) -- (uc6);
    \draw[thick] (-3.5,0.5) -- (uc7);
    \draw[thick] (-3.5,0.5) -- (uc8);
    \draw[thick] (-3.5,0.5) -- (uc9);
    
    \draw[thick] (-3.5,-4.3) -- (uc10);
    \draw[thick] (-3.5,-4.3) -- (uc11);
\end{tikzpicture}
\caption{Діаграма прецедентів інформаційної системи ательє}
\end{figure}

\textbf{Опис основних прецедентів:}

\begin{itemize}
    \item \textbf{Оформити замовлення} --- клієнт або адміністратор створює нове замовлення, вказуючи необхідні послуги, матеріали та терміни виконання;
    \item \textbf{Керування замовленнями} --- адміністратор може створювати, редагувати та видаляти замовлення, призначати виконавців;
    \item \textbf{Оновити статус} --- кравець змінює статус замовлення по мірі виконання роботи;
    \item \textbf{Формування звітів} --- адміністратор формує різні види звітів для аналізу діяльності.
\end{itemize}

\newpage
\subsection{Діаграма класів}

Діаграма класів відображає структуру системи, основні класи та зв'язки між ними.

\begin{figure}[h!]
\centering
\begin{tikzpicture}[scale=0.75, every node/.style={scale=0.75}]
    % Клас Client
    \draw[thick] (0,10) rectangle (3.5,13);
    \node at (1.75,12.7) {\textbf{Client}};
    \draw[thick] (0,12.5) -- (3.5,12.5);
    \node[text width=3.2cm, align=left] at (1.75,11.9) {\small
        - clientId: int\\
        - firstName: String\\
        - lastName: String\\
        - phone: String\\
        - email: String
    };
    \draw[thick] (0,11.2) -- (3.5,11.2);
    \node[text width=3.2cm, align=left] at (1.75,10.5) {\small
        + getFullName()\\
        + getMeasurements()
    };
    
    % Клас Measurement
    \draw[thick] (0,6) rectangle (3.5,9);
    \node at (1.75,8.7) {\textbf{Measurement}};
    \draw[thick] (0,8.5) -- (3.5,8.5);
    \node[text width=3.2cm, align=left] at (1.75,7.9) {\small
        - measurementId: int\\
        - clientId: int\\
        - chest: float\\
        - waist: float\\
        - height: float
    };
    \draw[thick] (0,7.2) -- (3.5,7.2);
    \node[text width=3.2cm, align=left] at (1.75,6.5) {\small
        + validate()
    };
    
    % Клас Order
    \draw[thick] (6,10) rectangle (10,13.5);
    \node at (8,13.2) {\textbf{Order}};
    \draw[thick] (6,13) -- (10,13);
    \node[text width=3.7cm, align=left] at (8,12.1) {\small
        - orderId: int\\
        - clientId: int\\
        - workerId: int\\
        - orderDate: Date\\
        - dueDate: Date\\
        - status: String\\
        - totalPrice: decimal
    };
    \draw[thick] (6,11.2) -- (10,11.2);
    \node[text width=3.7cm, align=left] at (8,10.5) {\small
        + calculatePrice()\\
        + assignWorker()\\
        + updateStatus()
    };
    
    % Клас Service
    \draw[thick] (12,10) rectangle (15.5,13);
    \node at (13.75,12.7) {\textbf{Service}};
    \draw[thick] (12,12.5) -- (15.5,12.5);
    \node[text width=3.2cm, align=left] at (13.75,11.9) {\small
        - serviceId: int\\
        - name: String\\
        - description: String\\
        - basePrice: decimal\\
        - duration: int
    };
    \draw[thick] (12,11.2) -- (15.5,11.2);
    \node[text width=3.2cm, align=left] at (13.75,10.5) {\small
        + getPrice()
    };
    
    % Клас OrderItem
    \draw[thick] (6,6) rectangle (10,9);
    \node at (8,8.7) {\textbf{OrderItem}};
    \draw[thick] (6,8.5) -- (10,8.5);
    \node[text width=3.7cm, align=left] at (8,7.7) {\small
        - orderItemId: int\\
        - orderId: int\\
        - serviceId: int\\
        - quantity: int\\
        - price: decimal
    };
    \draw[thick] (6,7) -- (10,7);
    \node[text width=3.7cm, align=left] at (8,6.4) {\small
        + calculateTotal()
    };
    
    % Клас Worker
    \draw[thick] (12,6) rectangle (15.5,9);
    \node at (13.75,8.7) {\textbf{Worker}};
    \draw[thick] (12,8.5) -- (15.5,8.5);
    \node[text width=3.2cm, align=left] at (13.75,7.7) {\small
        - workerId: int\\
        - firstName: String\\
        - lastName: String\\
        - position: String\\
        - phone: String
    };
    \draw[thick] (12,7) -- (15.5,7);
    \node[text width=3.2cm, align=left] at (13.75,6.4) {\small
        + getActiveOrders()
    };
    
    % Клас Material
    \draw[thick] (0,2) rectangle (3.5,5);
    \node at (1.75,4.7) {\textbf{Material}};
    \draw[thick] (0,4.5) -- (3.5,4.5);
    \node[text width=3.2cm, align=left] at (1.75,3.7) {\small
        - materialId: int\\
        - name: String\\
        - unit: String\\
        - pricePerUnit: decimal\\
        - stockQuantity: float
    };
    \draw[thick] (0,3) -- (3.5,3);
    \node[text width=3.2cm, align=left] at (1.75,2.4) {\small
        + checkAvailability()
    };
    
    % Клас MaterialUsage
    \draw[thick] (6,2) rectangle (10,5);
    \node at (8,4.7) {\textbf{MaterialUsage}};
    \draw[thick] (6,4.5) -- (10,4.5);
    \node[text width=3.7cm, align=left] at (8,3.7) {\small
        - usageId: int\\
        - orderId: int\\
        - materialId: int\\
        - quantity: float\\
        - cost: decimal
    };
    \draw[thick] (6,3) -- (10,3);
    \node[text width=3.7cm, align=left] at (8,2.4) {\small
        + calculateCost()
    };
    
    % Зв'язки
    \draw[thick, ->] (1.75,10) -- (1.75,9) node[midway, right] {\small 1..N};
    \draw[thick, ->] (3.5,11.5) -- (6,11.5) node[midway, above] {\small 1..N};
    \draw[thick, ->] (8,10) -- (8,9) node[midway, right] {\small 1..N};
    \draw[thick, ->] (10,7.5) -- (12,7.5) node[midway, above] {\small N..1};
    \draw[thick, ->] (10,11.5) -- (12,11.5) node[midway, above] {\small N..M};
    \draw[thick, ->] (13.75,6) -- (13.75,4.5) -- (10,4.5) -- (10,9) node[near start, right] {\small 1..N};
    \draw[thick, ->] (8,6) -- (8,5) node[midway, right] {\small 1..N};
    \draw[thick, ->] (6,3.5) -- (3.5,3.5) node[midway, above] {\small N..1};
\end{tikzpicture}
\caption{Діаграма класів інформаційної системи}
\end{figure}

\newpage
\subsection{Модель бази даних}

Концептуальна модель бази даних представлена у вигляді ER-діаграми, яка відображає основні сутності та зв'язки між ними.

\begin{figure}[h!]
\centering
\begin{tikzpicture}[node distance=1cm, auto, scale=0.85, every node/.style={scale=0.85}]
    % Стилі без кольорів
    \tikzstyle{entity} = [rectangle, draw, thick, text width=2.8cm, text centered, minimum height=1.2cm]
    \tikzstyle{relationship} = [diamond, draw, thick, text width=2cm, text centered, minimum height=1cm, aspect=2]
    
    % Сутності
    \node[entity] (client) at (0,8) {КЛІЄНТ\\(CLIENT)};
    \node[entity] (order) at (6,8) {ЗАМОВЛЕННЯ\\(ORDER)};
    \node[entity] (worker) at (6,4) {ПРАЦІВНИК\\(WORKER)};
    \node[entity] (service) at (12,8) {ПОСЛУГА\\(SERVICE)};
    \node[entity] (material) at (0,4) {МАТЕРІАЛ\\(MATERIAL)};
    \node[entity] (measurement) at (0,0) {МІРКИ\\(MEASUREMENT)};
    
    % Зв'язки
    \node[relationship] (makes) at (3,8) {оформляє};
    \node[relationship] (executes) at (6,6) {виконує};
    \node[relationship] (includes) at (9,8) {включає};
    \node[relationship] (uses) at (3,4) {використовує};
    \node[relationship] (has) at (0,2) {має};
    
    % Лінії зв'язків
    \draw[thick] (client) -- (makes) node[near start, above] {\small 1};
    \draw[thick] (makes) -- (order) node[near end, above] {\small N};
    
    \draw[thick] (worker) -- (executes) node[near start, right] {\small 1};
    \draw[thick] (executes) -- (order) node[near end, right] {\small N};
    
    \draw[thick] (order) -- (includes) node[near start, above] {\small N};
    \draw[thick] (includes) -- (service) node[near end, above] {\small M};
    
    \draw[thick] (order.south) -- (6,4.8) -- (3,4.8) -- (uses) node[near end, above] {\small N};
    \draw[thick] (uses) -- (material) node[near end, above] {\small M};
    
    \draw[thick] (client.south) -- (0,6.5) -- (0,2.5) -- (has) node[near start, right] {\small 1};
    \draw[thick] (has) -- (measurement) node[near end, right] {\small N};
\end{tikzpicture}
\caption{ER-діаграма бази даних}
\end{figure}

\subsubsection{Опис таблиць бази даних}

\textbf{Таблиця CLIENT (Клієнти):}
\begin{itemize}
    \item client\_id (PK) --- унікальний ідентифікатор клієнта
    \item first\_name --- ім'я
    \item last\_name --- прізвище
    \item phone --- номер телефону
    \item email --- електронна пошта
    \item registration\_date --- дата реєстрації
    \item notes --- додаткові примітки
\end{itemize}

\textbf{Таблиця MEASUREMENT (Мірки):}
\begin{itemize}
    \item measurement\_id (PK) --- унікальний ідентифікатор
    \item client\_id (FK) --- посилання на клієнта
    \item chest --- обхват грудей
    \item waist --- обхват талії
    \item hips --- обхват стегон
    \item height --- зріст
    \item measurement\_date --- дата зняття мірок
\end{itemize}

\textbf{Таблиця WORKER (Працівники):}
\begin{itemize}
    \item worker\_id (PK) --- унікальний ідентифікатор
    \item first\_name --- ім'я
    \item last\_name --- прізвище
    \item position --- посада
    \item phone --- номер телефону
    \item hire\_date --- дата прийняття на роботу
    \item salary --- заробітна плата
\end{itemize}

\textbf{Таблиця SERVICE (Послуги):}
\begin{itemize}
    \item service\_id (PK) --- унікальний ідентифікатор
    \item name --- назва послуги
    \item description --- опис
    \item base\_price --- базова ціна
    \item duration\_hours --- тривалість виконання
    \item category --- категорія послуги
\end{itemize}

\textbf{Таблиця ORDER (Замовлення):}
\begin{itemize}
    \item order\_id (PK) --- унікальний ідентифікатор
    \item client\_id (FK) --- посилання на клієнта
    \item worker\_id (FK) --- посилання на виконавця
    \item order\_date --- дата оформлення
    \item due\_date --- термін виконання
    \item completion\_date --- дата завершення
    \item status --- статус (новий, в роботі, готовий, виданий)
    \item total\_price --- загальна вартість
    \item notes --- примітки
\end{itemize}

\textbf{Таблиця ORDER\_ITEM (Позиції замовлення):}
\begin{itemize}
    \item order\_item\_id (PK) --- унікальний ідентифікатор
    \item order\_id (FK) --- посилання на замовлення
    \item service\_id (FK) --- посилання на послугу
    \item quantity --- кількість
    \item price --- ціна
    \item description --- опис
\end{itemize}

\textbf{Таблиця MATERIAL (Матеріали):}
\begin{itemize}
    \item material\_id (PK) --- унікальний ідентифікатор
    \item name --- назва матеріалу
    \item unit --- одиниця виміру
    \item price\_per\_unit --- ціна за одиницю
    \item stock\_quantity --- кількість на складі
    \item min\_stock\_level --- мінімальний рівень запасів
    \item supplier --- постачальник
\end{itemize}

\textbf{Таблиця MATERIAL\_USAGE (Використання матеріалів):}
\begin{itemize}
    \item usage\_id (PK) --- унікальний ідентифікатор
    \item order\_id (FK) --- посилання на замовлення
    \item material\_id (FK) --- посилання на матеріал
    \item quantity --- кількість використаного матеріалу
    \item cost --- вартість
    \item usage\_date --- дата використання
\end{itemize}

\newpage
\subsection{Діаграма діяльності}

Діаграма діяльності для процесу оформлення та виконання замовлення.

\begin{figure}[h!]
\centering
\begin{tikzpicture}[node distance=1.3cm, scale=0.85, every node/.style={scale=0.85}]
    \tikzstyle{startstop} = [ellipse, draw, thick, minimum width=2cm, minimum height=0.8cm]
    \tikzstyle{process} = [rectangle, draw, thick, text width=3.5cm, text centered, minimum height=1cm]
    \tikzstyle{decision} = [diamond, draw, thick, text width=2cm, text centered, aspect=2, inner sep=0pt]
    \tikzstyle{arrow} = [thick, ->, >=stealth]
    
    % Вузли
    \node[startstop] (start) {Початок};
    \node[process, below of=start] (p1) {Прийом клієнта та консультація};
    \node[process, below of=p1] (p2) {Зняття мірок клієнта};
    \node[process, below of=p2] (p3) {Вибір послуги та матеріалів};
    \node[process, below of=p3] (p4) {Розрахунок вартості};
    \node[decision, below of=p4, yshift=-0.3cm] (d1) {Клієнт\\згоден?};
    \node[process, below of=d1, yshift=-0.7cm] (p5) {Оформлення замовлення};
    \node[process, below of=p5] (p6) {Призначення виконавця};
    \node[process, below of=p6] (p7) {Виконання замовлення};
    \node[process, below of=p7] (p8) {Примірка};
    \node[decision, below of=p8, yshift=-0.3cm] (d2) {Потрібні\\зміни?};
    \node[process, below of=d2, yshift=-0.7cm] (p9) {Видача замовлення};
    \node[process, below of=p9] (p10) {Розрахунок з клієнтом};
    \node[startstop, below of=p10] (end) {Кінець};
    
    \node[process, right of=d1, xshift=3.5cm] (p11) {Коригування умов};
    \node[process, right of=d2, xshift=3.5cm] (p12) {Внесення змін};
    
    % Стрілки
    \draw[arrow] (start) -- (p1);
    \draw[arrow] (p1) -- (p2);
    \draw[arrow] (p2) -- (p3);
    \draw[arrow] (p3) -- (p4);
    \draw[arrow] (p4) -- (d1);
    \draw[arrow] (d1) -- node[right] {\small так} (p5);
    \draw[arrow] (d1) -- node[above] {\small ні} (p11);
    \draw[arrow] (p11.north) |- (p3.east);
    \draw[arrow] (p5) -- (p6);
    \draw[arrow] (p6) -- (p7);
    \draw[arrow] (p7) -- (p8);
    \draw[arrow] (p8) -- (d2);
    \draw[arrow] (d2) -- node[right] {\small ні} (p9);
    \draw[arrow] (d2) -- node[above] {\small так} (p12);
    \draw[arrow] (p12.north) |- (p7.east);
    \draw[arrow] (p9) -- (p10);
    \draw[arrow] (p10) -- (end);
\end{tikzpicture}
\caption{Діаграма діяльності: процес оформлення та виконання замовлення}
\end{figure}

\newpage
\subsection{Архітектура системи}

Інформаційна система управління ательє побудована за трирівневою архітектурою:

\begin{figure}[h!]
\centering
\begin{tikzpicture}[node distance=2.5cm, scale=0.9, every node/.style={scale=0.9}]
    \tikzstyle{layer} = [rectangle, draw, thick, text width=10cm, text centered, minimum height=1.8cm]
    
    \node[layer] (presentation) {
        \textbf{Рівень представлення (Presentation Layer)}\\
        \vspace{0.2cm}
        Веб-інтерфейс / Десктоп додаток
    };
    
    \node[layer, below of=presentation] (business) {
        \textbf{Рівень бізнес-логіки (Business Logic Layer)}\\
        \vspace{0.2cm}
        Обробка замовлень, управління клієнтами, розрахунки, звіти
    };
    
    \node[layer, below of=business] (data) {
        \textbf{Рівень даних (Data Access Layer)}\\
        \vspace{0.2cm}
        База даних (MySQL / PostgreSQL)
    };
    
    \draw[thick, <->, >=stealth] (presentation) -- (business);
    \draw[thick, <->, >=stealth] (business) -- (data);
\end{tikzpicture}
\caption{Трирівнева архітектура системи}
\end{figure}

\textbf{Опис рівнів:}

\begin{enumerate}
    \item \textbf{Рівень представлення:} забезпечує взаємодію користувача з системою через графічний інтерфейс. Може бути реалізований як веб-додаток або десктоп-програма.
    
    \item \textbf{Рівень бізнес-логіки:} містить основну логіку роботи системи, обробляє запити користувачів, виконує валідацію даних, здійснює розрахунки та формує звіти.
    
    \item \textbf{Рівень даних:} відповідає за зберігання та отримання даних з бази даних, забезпечує цілісність та безпеку інформації.
\end{enumerate}

\subsubsection{Діаграма компонентів}

\begin{figure}[h!]
\centering
\begin{tikzpicture}[node distance=2cm, scale=0.85, every node/.style={scale=0.85}]
    \tikzstyle{component} = [rectangle, draw, thick, text width=2.8cm, text centered, minimum height=1.2cm]
    
    % Компоненти інтерфейсу
    \node[component] (ui1) at (0,7) {Модуль\\керування\\клієнтами};
    \node[component] (ui2) at (3.5,7) {Модуль\\замовлень};
    \node[component] (ui3) at (7,7) {Модуль\\складу};
    \node[component] (ui4) at (10.5,7) {Модуль\\звітів};
    
    % Бізнес-логіка
    \node[component] (bl1) at (1.75,4) {Обробка\\замовлень};
    \node[component] (bl2) at (5.25,4) {Управління\\даними};
    \node[component] (bl3) at (8.75,4) {Формування\\звітів};
    
    % Доступ до даних
    \node[component, text width=6cm] (dal) at (5.25,1.5) {Рівень доступу до даних (DAL)};
    
    % База даних
    \node[component, text width=6cm] (db) at (5.25,-1) {База даних};
    
    % Зв'язки
    \draw[thick, ->] (ui1) -- (bl1);
    \draw[thick, ->] (ui2) -- (bl1);
    \draw[thick, ->] (ui2) -- (bl2);
    \draw[thick, ->] (ui3) -- (bl2);
    \draw[thick, ->] (ui4) -- (bl3);
    
    \draw[thick, ->] (bl1) -- (dal);
    \draw[thick, ->] (bl2) -- (dal);
    \draw[thick, ->] (bl3) -- (dal);
    
    \draw[thick, <->, >=stealth] (dal) -- (db);
\end{tikzpicture}
\caption{Діаграма компонентів системи}
\end{figure}

\newpage
\subsection{Діаграма послідовності}

Діаграма послідовності для процесу створення нового замовлення.

\begin{figure}[h!]
\centering
\begin{tikzpicture}[scale=0.9, every node/.style={scale=0.8}]
    \tikzstyle{object} = [rectangle, draw, thick, text width=2.2cm, text centered, minimum height=0.8cm]
    
    % Об'єкти
    \node[object] (admin) at (0,0) {Адміністратор};
    \node[object] (ui) at (3,0) {UI};
    \node[object] (controller) at (6,0) {Controller};
    \node[object] (service) at (9,0) {Service};
    \node[object] (db) at (12,0) {Database};
    
    % Лінії життя
    \draw[dashed, thick] (admin) -- (0,-14);
    \draw[dashed, thick] (ui) -- (3,-14);
    \draw[dashed, thick] (controller) -- (6,-14);
    \draw[dashed, thick] (service) -- (9,-14);
    \draw[dashed, thick] (db) -- (12,-14);
    
    % Повідомлення
    \draw[->, thick] (0,-1) -- node[above, sloped] {\tiny введення даних} (3,-1.8);
    
    \draw[->, thick] (3,-2.2) -- node[above, sloped] {\tiny createOrder()} (6,-3);
    
    \draw[->, thick] (6,-3.4) -- node[above, sloped] {\tiny validateData()} (9,-4.2);
    \draw[<-, thick] (6,-4.6) -- node[below, sloped] {\tiny valid} (9,-4.6);
    
    \draw[->, thick] (6,-5.2) -- node[above, sloped] {\tiny getClient(id)} (9,-6);
    \draw[->, thick] (9,-6.4) -- node[above, sloped] {\tiny SELECT} (12,-7.2);
    \draw[<-, thick] (9,-7.6) -- node[below, sloped] {\tiny clientData} (12,-7.6);
    \draw[<-, thick] (6,-8) -- node[below, sloped] {\tiny client} (9,-8);
    
    \draw[->, thick] (6,-8.6) -- node[above, sloped] {\tiny saveOrder()} (9,-9.4);
    \draw[->, thick] (9,-9.8) -- node[above, sloped] {\tiny INSERT} (12,-10.6);
    \draw[<-, thick] (9,-11) -- node[below, sloped] {\tiny orderId} (12,-11);
    \draw[<-, thick] (6,-11.4) -- node[below, sloped] {\tiny success} (9,-11.4);
    
    \draw[<-, thick] (3,-12) -- node[below, sloped] {\tiny orderCreated} (6,-12);
    \draw[<-, thick] (0,-12.8) -- node[below, sloped] {\tiny підтвердження} (3,-13.4);
\end{tikzpicture}
\caption{Діаграма послідовності: створення замовлення}
\end{figure}

\newpage
\subsection{Діаграма станів}

Діаграма станів для сутності "Замовлення" відображає можливі стани замовлення та переходи між ними.

\begin{figure}[h!]
\centering
\begin{tikzpicture}[node distance=2.8cm, auto, scale=0.85, every node/.style={scale=0.85}]
    \tikzstyle{state} = [rectangle, rounded corners, draw, thick, text width=2.8cm, text centered, minimum height=1.2cm]
    \tikzstyle{initial} = [circle, draw, thick, fill=black, minimum size=0.4cm]
    \tikzstyle{final} = [circle, draw, thick, double, minimum size=0.6cm]
    
    % Стани
    \node[initial] (init) {};
    \node[state] (new) [right of=init, xshift=-1cm] {Новий};
    \node[state] (accepted) [below of=new] {Прийнято\\в роботу};
    \node[state] (inprogress) [right of=accepted, xshift=1.5cm] {В процесі\\виконання};
    \node[state] (fitting) [below of=accepted] {Примірка};
    \node[state] (ready) [right of=fitting, xshift=1.5cm] {Готовий};
    \node[state] (completed) [below of=fitting] {Виданий};
    \node[state] (cancelled) [left of=fitting, xshift=-1.5cm] {Скасовано};
    \node[final] (end) [below of=completed] {};
    
    % Переходи
    \draw[->, thick, >=stealth] (init) -- (new);
    \draw[->, thick, >=stealth] (new) -- node[right] {\tiny призначити} (accepted);
    \draw[->, thick, >=stealth] (accepted) -- node[above] {\tiny почати} (inprogress);
    \draw[->, thick, >=stealth] (inprogress) -- node[right] {\tiny завершити} (fitting);
    \draw[->, thick, >=stealth] (fitting) -- node[above] {\tiny OK} (ready);
    \draw[->, thick, >=stealth] (fitting.east) .. controls +(right:1.5cm) and +(right:1.5cm) .. node[right] {\tiny зміни} (inprogress.east);
    \draw[->, thick, >=stealth] (ready) -- (completed);
    \draw[->, thick, >=stealth] (completed) -- (end);
    \draw[->, thick, >=stealth] (new) -- node[above] {\tiny скасувати} (cancelled);
    \draw[->, thick, >=stealth] (accepted) -- (cancelled);
    \draw[->, thick, >=stealth] (cancelled) -- (end);
\end{tikzpicture}
\caption{Діаграма станів замовлення}
\end{figure}

\textbf{Опис станів:}
\begin{itemize}
    \item \textbf{Новий} --- замовлення створено, але ще не прийнято в роботу;
    \item \textbf{Прийнято в роботу} --- замовлення призначено виконавцю;
    \item \textbf{В процесі виконання} --- виконавець працює над замовленням;
    \item \textbf{Примірка} --- клієнт проводить примірку виробу;
    \item \textbf{Готовий} --- виріб готовий до видачі;
    \item \textbf{Виданий} --- замовлення видано клієнту;
    \item \textbf{Скасовано} --- замовлення скасовано клієнтом або адміністратором.
\end{itemize}

\newpage
\section{РЕАЛІЗАЦІЯ СИСТЕМИ}

\subsection{Технології та інструменти}

Для реалізації інформаційної системи управління ательє пропонується використовувати наступний стек технологій:

\textbf{Backend (серверна частина):}
\begin{itemize}
    \item Мова програмування: Java / Python / C\#
    \item Фреймворк: Spring Boot / Django / ASP.NET Core
    \item ORM: Hibernate / Django ORM / Entity Framework
    \item База даних: PostgreSQL / MySQL
\end{itemize}

\textbf{Frontend (клієнтська частина):}
\begin{itemize}
    \item HTML5, CSS3, JavaScript
    \item Фреймворк: React / Angular / Vue.js
    \item UI-бібліотека: Material-UI / Bootstrap
\end{itemize}

\textbf{Інструменти розробки:}
\begin{itemize}
    \item Система контролю версій: Git
    \item IDE: IntelliJ IDEA / Visual Studio Code / PyCharm
    \item Інструменти для тестування: JUnit / pytest / NUnit
    \item Документування API: Swagger / OpenAPI
\end{itemize}

\subsection{Структура проекту}

Приклад структури проекту для веб-додатку:

\begin{verbatim}
atelier-system/
├── backend/
│   ├── src/
│   │   ├── controllers/      # Контролери
│   │   ├── models/           # Моделі даних
│   │   ├── services/         # Бізнес-логіка
│   │   ├── repositories/     # Робота з БД
│   │   └── config/           # Конфігурація
│   └── tests/                # Тести
├── frontend/
│   ├── src/
│   │   ├── components/       # React компоненти
│   │   ├── pages/            # Сторінки
│   │   ├── services/         # API сервіси
│   │   └── utils/            # Утиліти
│   └── public/               # Статичні файли
└── database/
    ├── migrations/           # Міграції БД
    └── seeds/                # Початкові дані
\end{verbatim}

\newpage
\subsection{SQL-скрипти створення таблиць}

\begin{lstlisting}[language=SQL, caption=Створення таблиць бази даних, basicstyle=\small\ttfamily, breaklines=true, frame=single]
-- Tablycya klientiv
CREATE TABLE client (
    client_id SERIAL PRIMARY KEY,
    first_name VARCHAR(50) NOT NULL,
    last_name VARCHAR(50) NOT NULL,
    phone VARCHAR(20),
    email VARCHAR(100),
    registration_date DATE DEFAULT CURRENT_DATE,
    notes TEXT
);

-- Tablycya mirok
CREATE TABLE measurement (
    measurement_id SERIAL PRIMARY KEY,
    client_id INTEGER REFERENCES client(client_id),
    chest DECIMAL(5,2),
    waist DECIMAL(5,2),
    hips DECIMAL(5,2),
    height DECIMAL(5,2),
    measurement_date DATE DEFAULT CURRENT_DATE
);

-- Tablycya pratsivnykiv
CREATE TABLE worker (
    worker_id SERIAL PRIMARY KEY,
    first_name VARCHAR(50) NOT NULL,
    last_name VARCHAR(50) NOT NULL,
    position VARCHAR(50),
    phone VARCHAR(20),
    hire_date DATE,
    salary DECIMAL(10,2)
);

-- Tablycya posluh
CREATE TABLE service (
    service_id SERIAL PRIMARY KEY,
    name VARCHAR(100) NOT NULL,
    description TEXT,
    base_price DECIMAL(10,2),
    duration_hours INTEGER,
    category VARCHAR(50)
);
\end{lstlisting}

\newpage
\begin{lstlisting}[language=SQL, caption=Продовження створення таблиць, basicstyle=\small\ttfamily, breaklines=true, frame=single]
-- Tablycya zamovlen
CREATE TABLE "order" (
    order_id SERIAL PRIMARY KEY,
    client_id INTEGER REFERENCES client(client_id),
    worker_id INTEGER REFERENCES worker(worker_id),
    order_date DATE DEFAULT CURRENT_DATE,
    due_date DATE,
    completion_date DATE,
    status VARCHAR(20) DEFAULT 'new',
    total_price DECIMAL(10,2),
    notes TEXT
);

-- Tablycya pozytsiy zamovlennya
CREATE TABLE order_item (
    order_item_id SERIAL PRIMARY KEY,
    order_id INTEGER REFERENCES "order"(order_id),
    service_id INTEGER REFERENCES service(service_id),
    quantity INTEGER DEFAULT 1,
    price DECIMAL(10,2),
    description TEXT
);

-- Tablycya materialiv
CREATE TABLE material (
    material_id SERIAL PRIMARY KEY,
    name VARCHAR(100) NOT NULL,
    unit VARCHAR(20),
    price_per_unit DECIMAL(10,2),
    stock_quantity DECIMAL(10,2),
    min_stock_level DECIMAL(10,2),
    supplier VARCHAR(100)
);

-- Tablycya vykorystannya materialiv
CREATE TABLE material_usage (
    usage_id SERIAL PRIMARY KEY,
    order_id INTEGER REFERENCES "order"(order_id),
    material_id INTEGER REFERENCES material(material_id),
    quantity DECIMAL(10,2),
    cost DECIMAL(10,2),
    usage_date DATE DEFAULT CURRENT_DATE
);
\end{lstlisting}

\newpage
\subsection{Приклади коду}

\textbf{Клас Order (Java):}

\begin{lstlisting}[language=Java, caption=Клас Order, basicstyle=\small\ttfamily, breaklines=true, frame=single]
public class Order {
    private int orderId;
    private int clientId;
    private int workerId;
    private LocalDate orderDate;
    private LocalDate dueDate;
    private LocalDate completionDate;
    private OrderStatus status;
    private BigDecimal totalPrice;
    private String notes;
    
    public enum OrderStatus {
        NEW, ACCEPTED, IN_PROGRESS, FITTING, 
        READY, COMPLETED, CANCELLED
    }
    
    public BigDecimal calculateTotalPrice(
        List<OrderItem> items, 
        List<MaterialUsage> materials) {
        
        BigDecimal total = BigDecimal.ZERO;
        
        for (OrderItem item : items) {
            total = total.add(
                item.getPrice().multiply(
                    new BigDecimal(item.getQuantity())
                )
            );
        }
        
        for (MaterialUsage material : materials) {
            total = total.add(material.getCost());
        }
        
        return total;
    }
    
    public void updateStatus(OrderStatus newStatus) {
        if (this.status == OrderStatus.COMPLETED || 
            this.status == OrderStatus.CANCELLED) {
            throw new IllegalStateException(
                "Cannot change completed order status"
            );
        }
        this.status = newStatus;
        
        if (newStatus == OrderStatus.COMPLETED) {
            this.completionDate = LocalDate.now();
        }
    }
}
\end{lstlisting}

\newpage
\section{ТЕСТУВАННЯ ТА ВПРОВАДЖЕННЯ}

\subsection{План тестування}

Тестування інформаційної системи включає наступні етапи:

\begin{enumerate}
    \item \textbf{Модульне тестування (Unit Testing):}
    \begin{itemize}
        \item Тестування окремих методів та функцій
        \item Перевірка коректності обчислень
        \item Валідація вхідних даних
    \end{itemize}
    
    \item \textbf{Інтеграційне тестування:}
    \begin{itemize}
        \item Тестування взаємодії між модулями
        \item Перевірка роботи з базою даних
        \item Тестування API endpoints
    \end{itemize}
    
    \item \textbf{Системне тестування:}
    \begin{itemize}
        \item Тестування повного функціоналу системи
        \item Перевірка бізнес-процесів
        \item Тестування продуктивності
    \end{itemize}
    
    \item \textbf{Приймальне тестування:}
    \begin{itemize}
        \item Тестування користувачами
        \item Перевірка відповідності вимогам
        \item Виявлення зауважень
    \end{itemize}
\end{enumerate}

\subsection{Приклади тестових сценаріїв}

\textbf{Тестовий сценарій 1: Створення нового замовлення}

\begin{enumerate}
    \item Адміністратор входить в систему
    \item Вибирає клієнта зі списку або створює нового
    \item Додає послуги до замовлення
    \item Вказує необхідні матеріали
    \item Система розраховує загальну вартість
    \item Адміністратор вказує термін виконання
    \item Система зберігає замовлення
    \item Очікуваний результат: замовлення створено, статус "Новий"
\end{enumerate}

\textbf{Тестовий сценарій 2: Відстеження виконання замовлення}

\begin{enumerate}
    \item Кравець входить в систему
    \item Переглядає список своїх замовлень
    \item Вибирає замовлення в роботі
    \item Оновлює статус на "В процесі виконання"
    \item Після завершення змінює статус на "Готовий"
    \item Очікуваний результат: статус оновлено, клієнт може бути повідомлений
\end{enumerate}

\subsection{Впровадження системи}

Етапи впровадження:

\begin{enumerate}
    \item \textbf{Підготовчий етап:}
    \begin{itemize}
        \item Налаштування серверного обладнання
        \item Встановлення програмного забезпечення
        \item Створення бази даних
    \end{itemize}
    
    \item \textbf{Міграція даних:}
    \begin{itemize}
        \item Перенесення існуючих даних про клієнтів
        \item Імпорт каталогу послуг
        \item Завантаження довідників
    \end{itemize}
    
    \item \textbf{Навчання персоналу:}
    \begin{itemize}
        \item Проведення тренінгів для адміністраторів
        \item Навчання кравців роботі з системою
        \item Підготовка інструкцій користувача
    \end{itemize}
    
    \item \textbf{Пілотний запуск:}
    \begin{itemize}
        \item Робота в тестовому режимі
        \item Виявлення та усунення помилок
        \item Збір зворотного зв'язку
    \end{itemize}
    
    \item \textbf{Промислова експлуатація:}
    \begin{itemize}
        \item Повний перехід на нову систему
        \item Технічна підтримка
        \item Моніторинг роботи системи
    \end{itemize}
\end{enumerate}

\newpage
\section{ВИСНОВКИ}

В результаті виконання курсової роботи була розроблена інформаційна система управління ательє, яка дозволяє автоматизувати основні бізнес-процеси підприємства.

\textbf{Основні результати роботи:}

\begin{enumerate}
    \item Проведено аналіз предметної області та виявлено основні бізнес-процеси ательє, що потребують автоматизації.
    
    \item Визначено функціональні вимоги до інформаційної системи, які включають управління клієнтами, замовленнями, матеріалами, персоналом та формування звітності.
    
    \item Розроблено діаграму прецедентів, що відображає взаємодію трьох типів користувачів з системою: клієнта, адміністратора та кравця.
    
    \item Створено діаграму класів, яка описує структуру системи та зв'язки між основними сутностями.
    
    \item Спроектовано модель бази даних, що складається з 8 таблиць та забезпечує зберігання всієї необхідної інформації.
    
    \item Розроблено діаграму діяльності для процесу оформлення та виконання замовлення, що демонструє послідовність дій.
    
    \item Описано трирівневу архітектуру системи, яка забезпечує розділення відповідальності та масштабованість.
    
    \item Розроблено діаграму станів для сутності "Замовлення", що визначає життєвий цикл замовлення.
    
    \item Підготовлено SQL-скрипти для створення таблиць бази даних та приклади коду основних класів.
\end{enumerate}

\textbf{Переваги розробленої системи:}

\begin{itemize}
    \item Автоматизація обліку клієнтів та їх замовлень
    \item Спрощення процесу оформлення та відстеження замовлень
    \item Контроль витрачання матеріалів та складських залишків
    \item Ефективний розподіл навантаження між працівниками
    \item Можливість формування аналітичних звітів
    \item Покращення якості обслуговування клієнтів
    \item Підвищення продуктивності роботи ательє
\end{itemize}

\textbf{Перспективи подальшого розвитку:}

\begin{itemize}
    \item Додавання мобільного додатку для клієнтів
    \item Інтеграція з онлайн-оплатою
    \item Впровадження системи лояльності для постійних клієнтів
    \item Додавання модуля для роботи з постачальниками
    \item Інтеграція з бухгалтерськими системами
    \item Впровадження аналітики на основі машинного навчання для прогнозування попиту
\end{itemize}

Розроблена інформаційна система є повноцінним інструментом для управління ательє та може бути впроваджена на підприємствах різного масштабу.

\newpage
\section{СПИСОК ВИКОРИСТАНИХ ДЖЕРЕЛ}

\begin{enumerate}
    \item Гамма Э., Хелм Р., Джонсон Р., Влиссидес Дж. Приемы объектно-ориентированного проектирования. Паттерны проектирования. --- СПб.: Питер, 2020. --- 368 с.
    
    \item Фаулер М. UML. Основы. --- 3-е изд. --- СПб.: Символ-Плюс, 2019. --- 192 с.
    
    \item Коннолли Т., Бегг К. Базы данных. Проектирование, реализация и сопровождение. Теория и практика. --- 3-е изд. --- М.: Вильямс, 2018. --- 1440 с.
    
    \item Буч Г., Рамбо Дж., Джекобсон А. Язык UML. Руководство пользователя. --- 2-е изд. --- М.: ДМК Пресс, 2019. --- 496 с.
    
    \item Вендров А.М. Проектирование программного обеспечения экономических информационных систем. --- М.: Финансы и статистика, 2018. --- 544 с.
    
    \item ДСТУ 3008-95. Документація. Звіти у сфері науки і техніки. Структура і правила оформлення. --- К.: Держстандарт України, 1995.
    
    \item Офіційна документація PostgreSQL [Електронний ресурс]. --- Режим доступу: https://www.postgresql.org/docs/
    
    \item Офіційна документація Spring Framework [Електронний ресурс]. --- Режим доступу: https://spring.io/projects/spring-framework
    
    \item Sommerville I. Software Engineering. --- 10th ed. --- Pearson, 2019. --- 816 p.
    
    \item Pressman R.S., Maxim B.R. Software Engineering: A Practitioner's Approach. --- 9th ed. --- McGraw-Hill Education, 2020. --- 976 p.
\end{enumerate}

\newpage
\appendix
\section{ДОДАТОК А. Діаграма розгортання}

\begin{figure}[h!]
\centering
\begin{tikzpicture}[node distance=2.5cm, scale=0.85, every node/.style={scale=0.85}]
    \tikzstyle{node} = [rectangle, draw, thick, text width=3.2cm, text centered, minimum height=1.5cm]
    \tikzstyle{component} = [rectangle, draw, thick, text width=2.8cm, text centered, minimum height=0.8cm]
    
    % Вузли
    \node[node] (client) {
        \textbf{Клієнтський}\\
        \textbf{комп'ютер}\\
        \vspace{0.1cm}
        \small{Windows/Linux}
    };
    
    \node[node, below of=client, yshift=-0.5cm] (webserver) {
        \textbf{Веб-сервер}\\
        \vspace{0.1cm}
        \small{Apache/Nginx}
    };
    
    \node[node, left of=webserver, xshift=-2cm] (appserver) {
        \textbf{Сервер додатків}\\
        \vspace{0.1cm}
        \small{Tomcat/Node.js}
    };
    
    \node[node, right of=webserver, xshift=2cm] (dbserver) {
        \textbf{Сервер БД}\\
        \vspace{0.1cm}
        \small{PostgreSQL}
    };
    
    % Компоненти
    \node[component, below of=client, yshift=1.3cm] (browser) {Веб-браузер};
    \node[component, below of=webserver, yshift=1.3cm] (frontend) {Frontend};
    \node[component, below of=appserver, yshift=1.3cm] (backend) {Backend API};
    \node[component, below of=dbserver, yshift=1.3cm] (database) {База даних};
    
    % Зв'язки
    \draw[thick, <->, >=stealth] (client) -- node[right] {\small HTTP/HTTPS} (webserver);
    \draw[thick, <->, >=stealth] (webserver) -- node[above] {\small REST API} (appserver);
    \draw[thick, <->, >=stealth] (appserver) -- node[above] {\small SQL} (dbserver);
\end{tikzpicture}
\caption{Діаграма розгортання системи}
\end{figure}

\section{ДОДАТОК Б. Словник термінів}

\begin{itemize}
    \item \textbf{Ательє} --- підприємство, що надає послуги з пошиття та ремонту одягу
    \item \textbf{API} (Application Programming Interface) --- інтерфейс програмування додатків
    \item \textbf{Backend} --- серверна частина додатку
    \item \textbf{Frontend} --- клієнтська частина додатку
    \item \textbf{CRUD} --- Create, Read, Update, Delete (базові операції з даними)
    \item \textbf{ER-діаграма} --- діаграма "сутність-зв'язок"
    \item \textbf{ORM} (Object-Relational Mapping) --- об'єктно-реляційне відображення
    \item \textbf{REST} (Representational State Transfer) --- архітектурний стиль для веб-сервісів
    \item \textbf{SQL} (Structured Query Language) --- мова структурованих запитів
    \item \textbf{UML} (Unified Modeling Language) --- уніфікована мова моделювання
\end{itemize}

\end{document}
