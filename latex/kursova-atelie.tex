% !TEX program = xelatex
\documentclass[14pt,a4paper]{extarticle}

% Ukrainian and DSTU-friendly setup
\usepackage{fontspec}
\usepackage{polyglossia}
\setdefaultlanguage{ukrainian}
\setotherlanguage{english}
% Use CMU fonts with Cyrillic support
\setmainfont{CMU Serif}
\setsansfont{CMU Sans Serif}
\setmonofont{CMU Typewriter Text}
\newfontfamily\cyrillicfont{CMU Serif}
\newfontfamily\cyrillicfontsf{CMU Sans Serif}
\newfontfamily\cyrillicfonttt{CMU Typewriter Text}

\usepackage[a4paper,margin=2.5cm]{geometry}
\usepackage{setspace}
\onehalfspacing
\usepackage{indentfirst}
\setlength{\parindent}{1.25cm}
\setlength{\parskip}{0pt}
\usepackage{enumitem}
\setlist{nosep}
\usepackage{hyperref}
\hypersetup{colorlinks=true,linkcolor=black,urlcolor=black,citecolor=black}
\usepackage{graphicx}
\usepackage{caption}
\usepackage{subcaption}
\usepackage{longtable}
\usepackage{booktabs}
\usepackage{array}
\usepackage{float}
\usepackage{tocloft}
\usepackage{titling}
\usepackage{csquotes}
\usepackage{tikz}
\usetikzlibrary{arrows.meta,positioning,shapes,shapes.geometric,calc}
\usepackage{amsmath}
\usepackage{pgfplots}
\pgfplotsset{compat=1.18}
\usepackage{pgf-pie}
\usepackage{pgfgantt}
\usepackage{listings}
\usepackage{listingsutf8}

% Listings setup for C# and SQL
\lstdefinelanguage{CSharp}{
  morekeywords={abstract,event,new,struct,as,explicit,null,switch,base,extern,object,this,bool,false,operator,throw,break,finally,out,true,byte,fixed,override,try,case,float,params,typeof,catch,for,private,uint,char,foreach,protected,ulong,checked,goto,public,unchecked,class,if,readonly,unsafe,const,implicit,ref,ushort,continue,in,return,using,decimal,int,sbyte,virtual,default,interface,sealed,volatile,delegate,internal,short,void,do,is,sizeof,while,double,lock,stackalloc,else,long,static,enum,namespace,string},
  sensitive=true,
  morecomment=[l]{//},
  morecomment=[s]{/*}{*/},
  morestring=[b]"
}

\lstdefinestyle{code}{
  basicstyle=\ttfamily\small,
  keywordstyle=\color{blue},
  commentstyle=\color{gray},
  stringstyle=\color{teal},
  numbers=left,
  numberstyle=\tiny,
  stepnumber=1,
  numbersep=6pt,
  frame=single,
  showstringspaces=false,
  breaklines=true,
  tabsize=2,
  columns=fullflexible,
}
\lstset{style=code,inputencoding=utf8}

% Title page variables (edit as needed)
\newcommand{\University}{Назва закладу вищої освіти}
\newcommand{\Faculty}{Факультет/Інститут}
\newcommand{\Department}{Кафедра}
\newcommand{\WorkType}{Курсова робота}
\newcommand{\Topic}{Інформаційна система ательє}
\newcommand{\Discipline}{З дисципліни: Об'єктно-орієнтоване проєктування}
\newcommand{\Student}{Прізвище Ім'я По батькові}
\newcommand{\Group}{Група ХХ-ХХ}
\newcommand{\Supervisor}{Науковий керівник, ст. посада, ПІБ}
\newcommand{\City}{Місто}
\newcommand{\Year}{2025}

% Title page per DSTU
\pretitle{\begin{center}\large \University\\ \Faculty\\ \Department\\[2cm] \WorkType \\[0.5cm] \textbf{\Topic}\\[0.5cm] \Discipline \\[3cm]}
\posttitle{\par\end{center}}
\preauthor{\begin{flushright}}
\postauthor{\end{flushright}}
\author{\begin{tabular}{@{}l@{}}Виконав(ла): \\ студент(ка) \Group \\ \Student \\[0.5cm] Керівник: \\ \Supervisor\end{tabular}}
\date{\vfill \begin{center}\City~\Year\end{center}}

% TOC formatting
\renewcommand{\cftsecleader}{\cftdotfill{\cftdotsep}}
\setcounter{tocdepth}{2}

% Captions in Ukrainian
\addto\captionsukrainian{\renewcommand{\figurename}{Рис.}}
\addto\captionsukrainian{\renewcommand{\tablename}{Табл.}}
\addto\captionsukrainian{\renewcommand{\contentsname}{ЗМІСТ}}

\begin{document}

% Title page
\title{\WorkType: \Topic}
\maketitle
\thispagestyle{empty}
\clearpage

% TOC
\pagenumbering{roman}
\setcounter{page}{2}
\tableofcontents
\clearpage

% Abbreviations
\section*{Перелік умовних позначень, скорочень}
\addcontentsline{toc}{section}{Перелік умовних позначень, скорочень}
API — прикладний програмний інтерфейс; БД — база даних; UML — мова уніфікованого моделювання.
\clearpage

% Intro
\pagenumbering{arabic}
\section{Вступ}
Метою курсової роботи є розроблення та документування інформаційної системи ательє, що підтримує процеси прийому замовлень, обліку клієнтів, ведення номенклатури послуг та матеріалів, планування завантаження майстрів і розрахунку вартості робіт.

Актуальність теми обумовлена потребою малих підприємств сфери побутового обслуговування в цифровізації обліку та процесів, підвищенні прозорості взаємодії з клієнтами та оптимізації ресурсів.

Об'єкт дослідження — процеси діяльності ательє з пошиття та ремонту одягу. Предмет дослідження — методи аналізу вимог і проєктування програмних систем із використанням нотацій UML і ER.

Структура роботи відповідає вимогам ДСТУ та включає аналіз предметної області, формалізацію вимог, проєктування (діаграми прецедентів, активності, класів, компонентів, послідовностей, ER-діаграму), стислий опис реалізації, підходи до тестування та висновки.

\section{Аналіз предметної області та вимог}
\subsection{Опис предметної області}
Ательє надає послуги з пошиття та ремонту виробів. Основні сутності: клієнт, замовлення, виріб, послуга, матеріал, майстер, платіж. Замовлення має статуси: нове, у роботі, готове, видане, скасоване.

\subsection{Зацікавлені сторони}
\begin{itemize}
  \item Клієнт — оформлює замовлення, отримує виріб, здійснює оплату;
  \item Адміністратор — приймає замовлення, веде клієнтську базу, формує рахунки;
  \item Майстер — виконує роботи, фіксує етапи та витрати матеріалів;
  \item Власник — аналізує звітність, встановлює прайс-листи.
\end{itemize}

\subsection{Функціональні вимоги}
\begin{itemize}
  \item Реєстрація клієнтів та їх контактних даних;
  \item Оформлення замовлень з переліком робіт і матеріалів;
  \item Калькуляція вартості; облік платежів (передплата, остаточний розрахунок);
  \item Планування та відстеження етапів виконання робіт;
  \item Формування звітів (виручка, завантаженість майстрів, популярні послуги).
\end{itemize}

\subsection{Нефункціональні вимоги}
\begin{itemize}
  \item Зручність інтерфейсу для настільних браузерів;
  \item Збереження даних у реляційній БД; резервне копіювання;
  \item Аудит змін замовлень та платежів;
  \item Ролі та доступи (адміністратор, майстер, власник).
\end{itemize}

\section{Проєктування системи}
\subsection{Діаграма прецедентів}
На рис.~\ref{fig:usecase} подано узагальнену діаграму прецедентів.

\begin{figure}[H]
  \centering
  \begin{tikzpicture}[node distance=1.6cm]
    % Actors
    \node (client) [draw, rounded corners, align=center] {Актор\\Клієнт};
    \node (admin) [draw, rounded corners, align=center, below=of client] {Актор\\Адміністратор};
    \node (master) [draw, rounded corners, align=center, below=of admin] {Актор\\Майстер};
    % System boundary
    \node (sys) [draw, minimum width=9cm, minimum height=6cm, right=3.8cm of admin] {};
    \node at ($(sys.north)+(0,-0.4)$) {ІС Ательє};
    % Use cases
    \node (uc1) at ($(sys.west)!0.5!(sys.east)+(0,1.8)$) [draw, ellipse, align=center, minimum width=3.8cm] {Оформити\\замовлення};
    \node (uc2) [draw, ellipse, below=1.1cm of uc1, minimum width=3.8cm, align=center] {Розрахувати\\вартість};
    \node (uc3) [draw, ellipse, below=1.1cm of uc2, minimum width=3.8cm, align=center] {Виконати\\роботи};
    \node (uc4) [draw, ellipse, below=1.1cm of uc3, minimum width=3.8cm, align=center] {Облік\\платежів};
    % Associations
    \draw[-{Latex}] (client.east) -- (uc1.west);
    \draw[-{Latex}] (admin.east) -- (uc1.west);
    \draw[-{Latex}] (admin.east) |- (uc2.west);
    \draw[-{Latex}] (master.east) -- (uc3.west);
    \draw[-{Latex}] (admin.east) |- (uc4.west);
  \end{tikzpicture}
  \caption{Діаграма прецедентів для ІС ательє}
  \label{fig:usecase}
\end{figure}

\subsection{Діаграма активності: обробка замовлення}
\begin{figure}[H]
  \centering
  \begin{tikzpicture}[node distance=1.4cm]
    \tikzset{act/.style={draw,rounded corners,minimum width=3.6cm,align=center}};
    \node (start) [circle, fill=black, minimum size=6pt, inner sep=0pt] {};
    \node (a1) [act, below=of start] {Прийняти замовлення};
    \node (a2) [act, below=of a1] {Оцінити вартість \\ і терміни};
    \node (a3) [act, below=of a2] {Призначити майстра};
    \node (a4) [act, below=of a3] {Виконати роботи};
    \node (a5) [act, below=of a4] {Прийняти оплату \\ і видати виріб};
    \node (end) [circle, draw, minimum size=8pt, inner sep=0pt, below=of a5] {};
    \draw[-{Latex}] (start) -- (a1);
    \draw[-{Latex}] (a1) -- (a2);
    \draw[-{Latex}] (a2) -- (a3);
    \draw[-{Latex}] (a3) -- (a4);
    \draw[-{Latex}] (a4) -- (a5);
    \draw[-{Latex}] (a5) -- (end);
  \end{tikzpicture}
  \caption{Діаграма активності процесу замовлення}
  \label{fig:activity}
\end{figure}

\subsection{Діаграма класів}
\begin{figure}[H]
  \centering
  \begin{tikzpicture}[node distance=1.6cm]
    \tikzset{class/.style={draw,rectangle,minimum width=4cm,align=left}}
    \node (clientc) [class] {\textbf{Клієнт}\\\hrulefill\\+id: UUID\\+ПІБ: string\\+телефон: string};
    \node (order) [class, right=3.5cm of clientc] {\textbf{Замовлення}\\\hrulefill\\+id: UUID\\+дата: date\\+статус: enum\\+сума: money};
    \node (service) [class, below=of order] {\textbf{Послуга}\\\hrulefill\\+id: UUID\\+назва: string\\+ціна: money};
    \node (master) [class, below=of clientc] {\textbf{Майстер}\\\hrulefill\\+id: UUID\\+ПІБ: string\\+кваліфікація: string};

    % Relations
    \draw[-{Latex}] (clientc) -- node[above,sloped]{1..*} (order);
    \draw[-{Latex}] (order) -- node[right]{*} (service);
    \draw[-{Latex}] (master) -- node[above,sloped]{1..*} (order);
  \end{tikzpicture}
  \caption{Спрощена діаграма класів}
  \label{fig:class}
\end{figure}

\subsection{Діаграма послідовностей: оформлення замовлення}
\begin{figure}[H]
  \centering
  \begin{tikzpicture}[node distance=0.8cm]
    \tikzset{lifeline/.style={draw,minimum width=2.4cm,align=center}}
    \node (actor) [lifeline] {Клієнт};
    \node (ui) [lifeline, right=2.5cm of actor] {UI};
    \node (svc) [lifeline, right=2.5cm of ui] {Сервіс};
    \node (db) [lifeline, right=2.5cm of svc] {БД};

    % Lifelines
    \draw[dashed] (actor.south) -- ++(0,-5.5);
    \draw[dashed] (ui.south) -- ++(0,-5.5);
    \draw[dashed] (svc.south) -- ++(0,-5.5);
    \draw[dashed] (db.south) -- ++(0,-5.5);

    % Messages
    \draw[-{Latex}] (actor) -- node[above]{ввести дані} (ui);
    \draw[-{Latex}] (ui) -- node[above]{createOrder()} (svc);
    \draw[-{Latex}] (svc) -- node[above]{INSERT} (db);
    \draw[-{Latex}] (db) -- node[below]{OK} (svc);
    \draw[-{Latex}] (svc) -- node[below]{orderId} (ui);
    \draw[-{Latex}] (ui) -- node[below]{підтвердження} (actor);
  \end{tikzpicture}
  \caption{Діаграма послідовностей}
  \label{fig:sequence}
\end{figure}

\subsection{ER-діаграма бази даних}
\begin{figure}[H]
  \centering
  \begin{tikzpicture}[node distance=1.6cm]
    \tikzset{ent/.style={draw,rectangle,minimum width=3.8cm,align=center}};
    \tikzset{rel/.style={draw,diamond,aspect=2,align=center,inner sep=1pt}};

    \node (client) [ent] {CLIENT\\id, name, phone};
    \node (orderent) [ent, right=3.5cm of client] {ORDER\\id, date, status, sum, clientId};
    \node (serviceent) [ent, below=of orderent] {SERVICE\\id, title, price};
    \node (orderitem) [ent, below=of client] {ORDER\_ITEM\\orderId, serviceId, qty, price};

    \node (r1) [rel, right=1.6cm of client] {places};
    \node (r2) [rel, below=0.8cm of r1] {contains};

    \draw (client) -- node[above]{1} (r1);
    \draw (r1) -- node[above]{N} (orderent);
    \draw (orderent) -- node[right]{1} (r2);
    \draw (r2) -- node[right]{N} (orderitem);
    \draw (serviceent) -- node[right]{1} (r2);
  \end{tikzpicture}
  \caption{ER-діаграма БД}
  \label{fig:er}
\end{figure}

\subsection{Діаграма компонентів (структури)}
\begin{figure}[H]
  \centering
  \begin{tikzpicture}[node distance=1.6cm]
    \tikzset{comp/.style={draw,rectangle,minimum width=3.8cm,align=center}};
    \node (web) [comp] {Web UI};
    \node (api) [comp, right=3.5cm of web] {REST API};
    \node (dbcomp) [comp, right=3.5cm of api] {RDBMS};

    \draw[-{Latex}] (web) -- (api);
    \draw[-{Latex}] (api) -- (dbcomp);
  \end{tikzpicture}
  \caption{Діаграма компонентів}
  \label{fig:components}
\end{figure}

% --------------------------------------------------
% ДОДАТКОВІ РОЗДІЛИ ДЛЯ РОЗШИРЕННЯ ЗМІСТУ
% --------------------------------------------------
\section{Формалізація вимог}
\subsection{Функціональні вимоги}
Нижче наведено перелік основних функціональних вимог (FR). Для кожної вимоги вказано короткий опис, пріоритет і критерії приймання.

\setlength{\LTpre}{0pt}\setlength{\LTpost}{6pt}
\begin{longtable}{@{}p{1.6cm} p{4.2cm} p{7.2cm} p{1.7cm} p{6.0cm}@{}}
\toprule
\textbf{ID} & \textbf{Назва} & \textbf{Опис} & \textbf{Пріор.} & \textbf{Критерії приймання}\\
\midrule
\endfirsthead
\toprule
\textbf{ID} & \textbf{Назва} & \textbf{Опис} & \textbf{Пріор.} & \textbf{Критерії приймання}\\
\midrule
\endhead
FR-1 & Реєстрація клієнта & Система дозволяє створити клієнта з ПІБ і телефоном. & Високий & Після збереження клієнт відображається в переліку; валідація телефону. \\
FR-2 & Пошук клієнтів & Пошук клієнтів за ПІБ/телефоном з частковим збігом. & Високий & Пошук повертає релевантні записи за підрядком. \\
FR-3 & Створення замовлення & Оформлення нового замовлення для обраного клієнта. & Високий & Присвоєно ID, дата, статус=New; запис збережено. \\
FR-4 & Додавання послуг & Додавання позицій послуг до замовлення з ціною і кількістю. & Високий & Позиції зберігаються; загальна сума перерахована. \\
FR-5 & Калькуляція вартості & Розрахунок загальної вартості (роботи + матеріали). & Високий & Загальна сума відображається з точністю 2 знаки. \\
FR-6 & Редагування замовлення & Зміна складу, термінів, виконавця до старту робіт. & Середній & Зміни логуються; статусні обмеження дотримані. \\
FR-7 & Призначення майстра & Призначення відповідального майстра із графіка завантаження. & Високий & Враховано поточне навантаження; майстер отримує задачу. \\
FR-8 & Етапи виконання & Фіксація етапів: прийом, кроїння, пошиття, примірка, видача. & Середній & Хід відображається на шкалі прогресу замовлення. \\
FR-9 & Облік матеріалів & Списання матеріалів з партій та складу. & Середній & Залишки коректно зменшуються; не допускається від’ємний залишок. \\
FR-10 & Облік платежів & Фіксація передплати й остаточного розрахунку. & Високий & Сальдо=0 на момент видачі; квитанції формуються. \\
FR-11 & Скасування замовлення & Скасування з причиною; відкат бронювань, повернення коштів. & Середній & Лог змін; коректні фінансові проводки. \\
FR-12 & Видача виробу & Завершення робіт, зміна статусу на Delivered. & Високий & Статус змінено; дата видачі зафіксована. \\
FR-13 & Звіти по виручці & Формування звіту по місяцях і послугах. & Середній & Дані відповідають платежам і статусам. \\
FR-14 & Звіт по завантаженню & Звіт по зайнятості майстрів у періоді. & Середній & Відсоток завантаження обчислюється за задачами. \\
FR-15 & Прайс-лист & Управління переліком послуг і тарифами. & Високий & Історія змін цін зберігається. \\
FR-16 & Резервне копіювання & Експорт БД до файлу; імпорт з резерву. & Низький & Резерв створюється; відновлення повертає консистентний стан. \\
FR-17 & Ролі та доступ & Ролі: адміністратор, майстер, власник. & Високий & Доступи відповідають матриці прав. \\
FR-18 & Аудит & Лог змін замовлень і платежів. & Середній & Видно хто, коли, що змінив. \\
FR-19 & Повідомлення клієнту & Сповіщення (email/SMS) про етапи й готовність. & Низький & Шаблони листів; відстеження доставки. \\
FR-20 & Примірки & Планування і фіксація результатів примірок. & Середній & Коментарі зберігаються; історія доступна. \\
FR-21 & Фото виробу & Збереження фото дефектів/етапів для історії. & Низький & Фото пов’язані з замовленням; видимі в картці. \\
FR-22 & Знижки та промо & Застосування знижок і промокодів. & Низький & У чеку показана знижка; тотал коректний. \\
FR-23 & Податки & Нарахування ПДВ за ставкою. & Низький & Сума ПДВ обчислюється й відображається. \\
FR-24 & Повернення коштів & Оформлення повернення з підставою. & Низький & Коректні проводки; статус платежу оновлено. \\
FR-25 & Експорт звітів & Експорт у CSV/XLSX основних звітів. & Низький & Формат валідний; кодування UTF-8. \\
\bottomrule
\end{longtable}

\subsection{Нефункціональні вимоги}
\begin{longtable}{@{}p{1.8cm} p{5.0cm} p{10.0cm} p{2.0cm}@{}}
\toprule
\textbf{ID} & \textbf{Атрибут якості} & \textbf{Опис} & \textbf{Пріор.}\\
\midrule
\endfirsthead
\toprule
\textbf{ID} & \textbf{Атрибут якості} & \textbf{Опис} & \textbf{Пріор.}\\
\midrule
\endhead
NFR-1 & Продуктивність & 95-й перцентиль відповіді API $<$ 300\,мс для типових операцій. & Високий \\
NFR-2 & Доступність & 99{.}5\% на місяць; відновлення після збою $<$ 15\,хв. & Високий \\
NFR-3 & Безпека & Паролі хешуються (Argon2/BCrypt); JWT; ролева авторизація. & Високий \\
NFR-4 & Надійність даних & Транзакції з ACID; регулярні резервні копії. & Високий \\
NFR-5 & Масштабованість & Горизонтальне масштабування API; БД~— вертикальне. & Середній \\
NFR-6 & Супровідність & Документація OpenAPI; логування та трасування. & Середній \\
NFR-7 & UX & Послідовний дизайн; доступність WCAG 2.1 AA. & Середній \\
NFR-8 & Сумісність & Підтримка останніх версій Chrome/Firefox/Edge. & Середній \\
NFR-9 & Локалізація & Українська локаль за замовчуванням; підтримка i18n. & Низький \\
NFR-10 & Аудит & Історія змін критичних сутностей (замовлення, платежі). & Високий \\
\bottomrule
\end{longtable}

\subsection{Опис ключових прецедентів}
\paragraph{UC-1: Оформити замовлення} Мета — створити замовлення для клієнта, визначити перелік робіт, розрахувати вартість і терміни. Основний потік: вибір клієнта → введення виробу й послуг → розрахунок → збереження. Винятки: невалідні дані; недоступні матеріали.

\paragraph{UC-2: Виконати роботи} Мета — забезпечити виконання етапів з фіксацією часу та витрат. Потік: призначення майстра → етапи → контроль якості → готовність.

\paragraph{UC-3: Прийняти оплату} Мета — облік передплати/остаточного розрахунку. Потік: генерація рахунку → прийом платежу → оновлення сальдо.

\clearpage

\section{Модель даних і словник даних}
Наведено словник даних (Data Dictionary) для основних таблиць. Типи подано для SQLite; у промисловій СУБД можуть відрізнятися.

\begin{longtable}{@{}p{2.8cm} p{3.2cm} p{3.2cm} p{2.0cm} p{3.2cm} p{8.2cm}@{}}
\toprule
\textbf{Таблиця} & \textbf{Поле} & \textbf{Тип} & \textbf{Null} & \textbf{За замовч.} & \textbf{Опис}\\
\midrule
\endfirsthead
\toprule
\textbf{Таблиця} & \textbf{Поле} & \textbf{Тип} & \textbf{Null} & \textbf{За замовч.} & \textbf{Опис}\\
\midrule
\endhead
Client & Id & TEXT (UUID) & Ні & — & Ідентифікатор клієнта. \\
Client & FullName & TEXT & Ні & — & ПІБ клієнта. \\
Client & Phone & TEXT & Ні & — & Контактний телефон у міжнародному форматі. \\
\addlinespace
Service & Id & TEXT (UUID) & Ні & — & Ідентифікатор послуги. \\
Service & Title & TEXT & Ні & — & Назва послуги. \\
Service & Price & REAL & Ні & 0 & Базова ціна. \\
\addlinespace
\texttt{[Order]} & Id & TEXT (UUID) & Ні & — & Ідентифікатор замовлення. \\
\texttt{[Order]} & Date & TEXT (ISO) & Ні & now & Дата створення. \\
\texttt{[Order]} & Status & INTEGER & Ні & 0 & Статус (enum). \\
\texttt{[Order]} & Total & REAL & Ні & 0 & Загальна сума. \\
\texttt{[Order]} & ClientId & TEXT & Ні & — & Посилання на Client.Id. \\
\addlinespace
OrderItem & OrderId & TEXT & Ні & — & Посилання на Order.Id. \\
OrderItem & ServiceId & TEXT & Ні & — & Посилання на Service.Id. \\
OrderItem & Qty & REAL & Ні & 1 & Кількість. \\
OrderItem & Price & REAL & Ні & — & Ціна за одиницю. \\
\addlinespace
Payment & Id & TEXT (UUID) & Ні & — & Ідентифікатор платежу. \\
Payment & OrderId & TEXT & Ні & — & Посилання на Order.Id. \\
Payment & Date & TEXT (ISO) & Ні & now & Дата платежу. \\
Payment & Amount & REAL & Ні & — & Сума платежу. \\
Payment & Method & TEXT & Ні & cash & Метод (cash/card). \\
\addlinespace
Master & Id & TEXT (UUID) & Ні & — & Ідентифікатор майстра. \\
Master & FullName & TEXT & Ні & — & ПІБ майстра. \\
Master & Skill & TEXT & Так & — & Кваліфікація/спеціалізація. \\
\addlinespace
Material & Id & TEXT (UUID) & Ні & — & Ідентифікатор матеріалу. \\
Material & Title & TEXT & Ні & — & Назва матеріалу. \\
Material & Qty & REAL & Ні & 0 & Залишок на складі. \\
Material & Unit & TEXT & Ні & pcs & Одиниця виміру (м, м², pcs). \\
\addlinespace
AuditLog & Id & INTEGER & Ні & autoinc & Первинний ключ. \\
AuditLog & Entity & TEXT & Ні & — & Назва сутності. \\
AuditLog & EntityId & TEXT & Ні & — & Ідентифікатор сутності. \\
AuditLog & Action & TEXT & Ні & — & Дія (CREATE/UPDATE/DELETE). \\
AuditLog & ChangedAt & TEXT (ISO) & Ні & now & Час зміни. \\
AuditLog & UserId & TEXT & Так & — & Виконавець (якщо автентифікований). \\
\bottomrule
\end{longtable}

\subsection{Нормалізація та обмеження цілісності}
Схема відповідає щонайменше 3НФ: довідники винесено окремо, \mbox{Order-OrderItem} — зв’язок \(1\!:\!N\), суми та ПДВ — похідні поля. Референційна цілісність забезпечується зовнішніми ключами; тотал замовлення перераховується тригерами після змін позицій.

\clearpage

\section{Додаткові діаграми}
\subsection{Діаграма станів замовлення}
\begin{figure}[H]
  \centering
  \begin{tikzpicture}[node distance=2.2cm]
    \tikzset{state/.style={draw,rounded corners,minimum width=3.4cm,align=center,inner sep=4pt}};
    \node (new) [state] {New};
    \node (inprog) [state, right=of new] {InProgress};
    \node (ready) [state, right=of inprog] {Ready};
    \node (deliv) [state, right=of ready] {Delivered};
    \node (cancel) [state, below=of inprog] {Cancelled};
    \draw[-{Latex}] (new) -- node[above]{start} (inprog);
    \draw[-{Latex}] (inprog) -- node[above]{finish} (ready);
    \draw[-{Latex}] (ready) -- node[above]{issue} (deliv);
    \draw[-{Latex}] (new) -- node[left]{cancel} (cancel);
    \draw[-{Latex}] (inprog) -- node[right]{cancel} (cancel);
  \end{tikzpicture}
  \caption{Станова діаграма життєвого циклу замовлення}
\end{figure}

\subsection{Діаграма розгортання}
\begin{figure}[H]
  \centering
  \begin{tikzpicture}[node distance=2.4cm]
    \tikzset{nodebox/.style={draw,rectangle,minimum width=4.2cm,minimum height=1.2cm,align=center}};
    \node (browser) [nodebox] {Клієнтський браузер};
    \node (web) [nodebox, right=of browser] {Веб-сервер (Nginx)};
    \node (api) [nodebox, right=of web] {API (.NET)};
    \node (db) [nodebox, below=of api] {БД (SQLite/PostgreSQL)};
    \draw[-{Latex}] (browser) -- node[above]{HTTPS} (web);
    \draw[-{Latex}] (web) -- node[above]{HTTPS/REST} (api);
    \draw[-{Latex}] (api) -- node[right]{TCP} (db);
  \end{tikzpicture}
  \caption{Схема розгортання компонентів}
\end{figure}

\clearpage

\section{План-графік робіт}
\begin{center}
\begin{ganttchart}[
  time slot unit=month,
  x unit=0.65cm,
  y unit chart=0.7cm,
  vgrid,hgrid,
  title height=1,
  bar height=0.6
]{2025-01}{2025-06}
  \gantttitlecalendar{year, month} \\
  \ganttbar{Аналіз вимог}{2025-01}{2025-02} \\
  \ganttbar{Проєктування}{2025-02}{2025-03} \\
  \ganttbar{Реалізація}{2025-03}{2025-05} \\
  \ganttbar{Тестування}{2025-05}{2025-06} \\
  \ganttbar{Впровадження}{2025-06}{2025-06}
\end{ganttchart}
\end{center}

\clearpage

\section{Аналітика та візуалізація}
\subsection{Виручка за місяцями}
\begin{figure}[H]
  \centering
  \begin{tikzpicture}
    \begin{axis}[
      width=0.95\textwidth,height=6cm,
      xlabel={Місяць}, ylabel={Виручка, тис. грн},
      ymin=0, xtick=data,
      symbolic x coords={Січ,Лют,Бер,Кві,Тра,Чер,Лип,Сер,Вер,Жов,Лис,Гру},
      ymajorgrids
    ]
      \addplot+[mark=*, very thick, color=blue] coordinates {
        (Січ,180) (Лют,190) (Бер,210) (Кві,240) (Тра,260) (Чер,255)
        (Лип,270) (Сер,265) (Вер,280) (Жов,300) (Лис,295) (Гру,320)
      };
    \end{axis}
  \end{tikzpicture}
  \caption{Місячна виручка протягом року}
\end{figure}

\subsection{Завантаженість майстрів (стек-бар)}
\begin{figure}[H]
  \centering
  \begin{tikzpicture}
    \begin{axis}[
      ybar stacked,
      bar width=10pt,
      width=0.95\textwidth,height=6.2cm,
      ymin=0, ymax=160,
      ylabel={Години/міс.},
      symbolic x coords={Бер,Кві,Тра,Чер,Лип,Сер},
      xtick=data, legend columns=3, legend style={at={(0.5,-0.18)},anchor=north}
    ]
      \addplot coordinates {(Бер,20) (Кві,25) (Тра,28) (Чер,30) (Лип,26) (Сер,24)};\addlegendentry{Майстер A}
      \addplot coordinates {(Бер,18) (Кві,20) (Тра,22) (Чер,24) (Лип,20) (Сер,19)};\addlegendentry{Майстер B}
      \addplot coordinates {(Бер,15) (Кві,18) (Тра,20) (Чер,22) (Лип,21) (Сер,20)};\addlegendentry{Майстер C}
      \addplot coordinates {(Бер,12) (Кві,14) (Тра,16) (Чер,18) (Лип,15) (Сер,14)};\addlegendentry{Майстер D}
    \end{axis}
  \end{tikzpicture}
  \caption{Навантаження майстрів по місяцях}
\end{figure}

\clearpage

\section{Реалізація}
Програмна реалізація може бути виконана у вигляді веб-додатка з архітектурою клієнт–сервер. Клієнтська частина (наприклад, React) взаємодіє з серверним REST API (наприклад, Node.js/Express, Java Spring або Python FastAPI), дані зберігаються у реляційній СУБД (PostgreSQL/MySQL). Аутентифікація може базуватися на JWT, авторизація — на ролях.

\subsection{Фрагменти коду C\#}
\subsubsection*{DTO та моделі}
\begin{lstlisting}[language=CSharp,caption={Моделі домену для ательє}]
public sealed class Client
{
    public Guid Id { get; init; }
    public string FullName { get; set; } = string.Empty;
    public string Phone { get; set; } = string.Empty;
}

public enum OrderStatus
{
    New,
    InProgress,
    Ready,
    Delivered,
    Cancelled
}

public sealed class Order
{
    public Guid Id { get; init; }
    public DateOnly Date { get; init; } = DateOnly.FromDateTime(DateTime.UtcNow);
    public OrderStatus Status { get; set; } = OrderStatus.New;
    public decimal Total { get; set; }
    public Guid ClientId { get; init; }
}

public sealed class Service
{
    public Guid Id { get; init; }
    public string Title { get; set; } = string.Empty;
    public decimal Price { get; set; }
}
\end{lstlisting}

\subsubsection*{Робота зі SQLite (ADO.NET)}
\begin{lstlisting}[language=CSharp,caption={Ініціалізація SQLite та запити}]
using System.Data;
using Microsoft.Data.Sqlite; // dotnet add package Microsoft.Data.Sqlite

const string ConnectionString = "Data Source=atelier.db";

using var connection = new SqliteConnection(ConnectionString);
await connection.OpenAsync();

// Створення таблиць
var createSql = @"
CREATE TABLE IF NOT EXISTS Client (
  Id TEXT PRIMARY KEY,
  FullName TEXT NOT NULL,
  Phone TEXT NOT NULL
);
CREATE TABLE IF NOT EXISTS Service (
  Id TEXT PRIMARY KEY,
  Title TEXT NOT NULL,
  Price REAL NOT NULL
);
CREATE TABLE IF NOT EXISTS [Order] (
  Id TEXT PRIMARY KEY,
  Date TEXT NOT NULL,
  Status INTEGER NOT NULL,
  Total REAL NOT NULL,
  ClientId TEXT NOT NULL REFERENCES Client(Id)
);
";
await new SqliteCommand(createSql, connection).ExecuteNonQueryAsync();

// Вставка клієнта параметризовано
var insertClient = new SqliteCommand(
    "INSERT INTO Client (Id, FullName, Phone) VALUES ($id, $name, $phone)", connection);
insertClient.Parameters.AddWithValue("$id", Guid.NewGuid().ToString());
insertClient.Parameters.AddWithValue("$name", "Іван Петренко");
insertClient.Parameters.AddWithValue("$phone", "+380501112233");
await insertClient.ExecuteNonQueryAsync();

// Вибірка з мапінгом у об'єкти
var clients = new List<Client>();
var select = new SqliteCommand("SELECT Id, FullName, Phone FROM Client", connection);
using (var reader = await select.ExecuteReaderAsync(CommandBehavior.CloseConnection))
{
    while (await reader.ReadAsync())
    {
        clients.Add(new Client
        {
            Id = Guid.Parse(reader.GetString(0)),
            FullName = reader.GetString(1),
            Phone = reader.GetString(2)
        });
    }
}
\end{lstlisting}

\subsubsection*{Мінімальний Web API (.NET)}
\begin{lstlisting}[language=CSharp,caption={Мінімальний ендпойнт для замовлень}]
var builder = WebApplication.CreateBuilder(args);
var app = builder.Build();

app.MapGet("/orders", async () =>
{
    using var connection = new SqliteConnection("Data Source=atelier.db");
    await connection.OpenAsync();
    using var cmd = new SqliteCommand("SELECT Id, Date, Status, Total, ClientId FROM [Order]", connection);
    var list = new List<Order>();
    using var reader = await cmd.ExecuteReaderAsync();
    while (await reader.ReadAsync())
    {
        list.Add(new Order
        {
            Id = Guid.Parse(reader.GetString(0)),
            Date = DateOnly.Parse(reader.GetString(1)),
            Status = (OrderStatus)reader.GetInt32(2),
            Total = (decimal)reader.GetDouble(3),
            ClientId = Guid.Parse(reader.GetString(4))
        });
    }
    return Results.Ok(list);
});

app.Run();
\end{lstlisting}

\subsection{SQL (SQLite) схеми та запити}
\begin{lstlisting}[language=SQL,caption={DDL і базові запити для SQLite}]
-- Створення таблиць
CREATE TABLE IF NOT EXISTS Client (
  Id TEXT PRIMARY KEY,
  FullName TEXT NOT NULL,
  Phone TEXT NOT NULL
);

CREATE TABLE IF NOT EXISTS Service (
  Id TEXT PRIMARY KEY,
  Title TEXT NOT NULL,
  Price REAL NOT NULL
);

CREATE TABLE IF NOT EXISTS [Order] (
  Id TEXT PRIMARY KEY,
  Date TEXT NOT NULL,
  Status INTEGER NOT NULL,
  Total REAL NOT NULL,
  ClientId TEXT NOT NULL REFERENCES Client(Id)
);

-- Топ-5 популярних послуг
SELECT s.Title, COUNT(*) AS Times
FROM [Order] o
JOIN OrderItem oi ON oi.OrderId = o.Id
JOIN Service s ON s.Id = oi.ServiceId
GROUP BY s.Title
ORDER BY Times DESC
LIMIT 5;
\end{lstlisting}

\section{Тестування}
Запропоновано рівні тестування: модульне (сервісні методи калькуляції, валідації), інтеграційне (взаємодія API та БД), системне (сценарії оформлення замовлення, оплати), приймальне (з користувачами). Для нефункціональних аспектів — навантажувальне тестування ключових ендпойнтів.

\subsection{Статистичні графіки}
\begin{figure}[H]
  \centering
  \begin{tikzpicture}
    \begin{axis}[
      ybar, bar width=14pt,
      ymin=0,
      ylabel={Кількість замовлень},
      symbolic x coords={Січ,Лют,Бер,Кві,Тра,Чер},
      xtick=data,
      width=0.9\textwidth,height=6cm,
      nodes near coords
    ]
      \addplot coordinates {(Січ,18) (Лют,22) (Бер,28) (Кві,32) (Тра,30) (Чер,35)};
    \end{axis}
  \end{tikzpicture}
  \caption{Динаміка замовлень за півріччя}
\end{figure}

\begin{figure}[H]
  \centering
  \begin{tikzpicture}
    \pie[radius=2.6, text=legend, color={blue!60,orange!70!yellow,green!60,red!60}]{
      40/Пошиття,
      25/Ремонт,
      20/Підгонка,
      15/Інше
    }
  \end{tikzpicture}
  \caption{Розподіл типів послуг}
\end{figure}

\section{Висновки}
Було проаналізовано предметну область ательє, уточнено вимоги та розроблено модель системи за допомогою UML і ER-діаграм. Запропоновано архітектуру рішення та підходи до тестування. Результати можуть бути використані як основа для подальшої детальної реалізації.

\section*{Список використаних джерел}
\addcontentsline{toc}{section}{Список використаних джерел}
\begin{thebibliography}{9}
\bibitem{fowler}
М. Фаулер. UML Distilled: A Brief Guide to the Standard Object Modeling Language. Addison-Wesley, 2003.

\bibitem{connolly}
Т. Конноллі, К. Бегг. Database Systems: A Practical Approach to Design, Implementation, and Management. Pearson, 2014.

\bibitem{dstu}
Вимоги до оформлення кваліфікаційних робіт за ДСТУ. Доступ: \url{https://example.edu/dstu-guidelines} (дата звернення: 24.10.2025).
\end{thebibliography}

\end{document}
