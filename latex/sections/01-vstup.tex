\section{Вступ}
Метою курсової роботи є розроблення та документування інформаційної системи ательє, що підтримує процеси прийому замовлень, обліку клієнтів, ведення номенклатури послуг та матеріалів, планування завантаження майстрів і розрахунку вартості робіт.

Актуальність теми обумовлена потребою малих підприємств сфери побутового обслуговування в цифровізації обліку та процесів, підвищенні прозорості взаємодії з клієнтами та оптимізації ресурсів.

Об'єкт дослідження — процеси діяльності ательє з пошиття та ремонту одягу. Предмет дослідження — методи аналізу вимог і проєктування програмних систем із використанням нотацій UML і ER.

Структура роботи відповідає вимогам ДСТУ та включає аналіз предметної області, формалізацію вимог, проєктування (діаграми прецедентів, активності, класів, компонентів, послідовностей, ER-діаграму), стислий опис реалізації, підходи до тестування та висновки.
