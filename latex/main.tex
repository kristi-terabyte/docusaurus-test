% !TEX program = lualatex
\documentclass[14pt,a4paper]{extarticle}

% Ukrainian and DSTU-friendly setup
\usepackage{fontspec}
\usepackage{polyglossia}
\setdefaultlanguage{ukrainian}
\setotherlanguage{english}
% Portability-friendly TeX Gyre fonts (Times/Arial/Courier compatible)
\setmainfont{TeX Gyre Termes}
\setsansfont{TeX Gyre Heros}
\setmonofont{TeX Gyre Cursor}

\usepackage[a4paper,margin=2.5cm]{geometry}
\usepackage{setspace}
\onehalfspacing
\usepackage{indentfirst}
\setlength{\parindent}{1.25cm}
\setlength{\parskip}{0pt}
\usepackage{enumitem}
\setlist{nosep}
\usepackage{hyperref}
\hypersetup{colorlinks=true,linkcolor=black,urlcolor=black,citecolor=black}
\usepackage{graphicx}
\usepackage{caption}
\usepackage{subcaption}
\usepackage{longtable}
\usepackage{booktabs}
\usepackage{array}
\usepackage{float}
\usepackage{tocloft}
\usepackage{titling}
\usepackage{csquotes}
\usepackage{pdfpages}
\usepackage{tikz}
\usetikzlibrary{arrows.meta,positioning,shapes,calc}

% Bibliography (DSTU-like numeric style via biblatex-gost)
\usepackage[backend=biber,style=gost-numeric,language=auto,autolang=other]{biblatex}
\addbibresource{references.bib}

% Title page variables (change to match your university DSTU requirements)
\newcommand{\University}{Назва закладу вищої освіти}
\newcommand{\Faculty}{Факультет/Інститут}
\newcommand{\Department}{Кафедра}
\newcommand{\WorkType}{Курсова робота}
\newcommand{\Topic}{Інформаційна система ательє}
\newcommand{\Discipline}{З дисципліни: Об'єктно-орієнтоване проєктування}
\newcommand{\Student}{Прізвище Ім'я По батькові}
\newcommand{\Group}{Група ХХ-ХХ}
\newcommand{\Supervisor}{Науковий керівник, ст. посада, ПІБ}
\newcommand{\City}{Місто}
\newcommand{\Year}{2025}

% Title page per DSTU
\pretitle{\begin{center}\large \University\\ \Faculty\\ \Department\\[2cm] \WorkType \\[0.5cm] \textbf{\Topic}\\[0.5cm] \Discipline \\[3cm]}
\posttitle{\par\end{center}}
\preauthor{\begin{flushright}}
\postauthor{\end{flushright}}
\author{\begin{tabular}{@{}l@{}}Виконав(ла): \\ студент(ка) \Group \\ \Student \\[0.5cm] Керівник: \\ \Supervisor\end{tabular}}
\date{\vfill \begin{center}\City~\Year\end{center}}

% Formatting TOC per DSTU rough requirements
\renewcommand{\cftsecleader}{\cftdotfill{\cftdotsep}}
\setcounter{tocdepth}{2}

% Figure and table captions in Ukrainian
\addto\captionsukrainian{\renewcommand{\figurename}{Рис.}}
\addto\captionsukrainian{\renewcommand{\tablename}{Табл.}}
\addto\captionsukrainian{\renewcommand{\contentsname}{ЗМІСТ}}

\begin{document}

% Title page
\maketitle
\thispagestyle{empty}
\clearpage

% ЗМІСТ (Table of contents)
\pagenumbering{roman}
\setcounter{page}{2}
\tableofcontents
\clearpage

% Перелік умовних позначень, скорочень (optional)
\section*{Перелік умовних позначень, скорочень}
\addcontentsline{toc}{section}{Перелік умовних позначень, скорочень}
API — прикладний програмний інтерфейс; БД — база даних; UML — мова уніфікованого моделювання.
\clearpage

% Вступ
\pagenumbering{arabic}
\section{Вступ}
Метою курсової роботи є розроблення та документування інформаційної системи ательє, що підтримує процеси прийому замовлень, обліку клієнтів, ведення номенклатури послуг та матеріалів, планування завантаження майстрів і розрахунку вартості робіт.

Актуальність теми обумовлена потребою малих підприємств сфери побутового обслуговування в цифровізації обліку та процесів, підвищенні прозорості взаємодії з клієнтами та оптимізації ресурсів.

Об'єкт дослідження — процеси діяльності ательє з пошиття та ремонту одягу. Предмет дослідження — методи аналізу вимог і проєктування програмних систем із використанням нотацій UML і ER.

Структура роботи відповідає вимогам ДСТУ та включає аналіз предметної області, формалізацію вимог, проєктування (діаграми прецедентів, активності, класів, компонентів, послідовностей, ER-діаграму), стислий опис реалізації, підходи до тестування та висновки.


% 1. Аналіз предметної області та вимог
\section{Аналіз предметної області та вимог}
\subsection{Опис предметної області}
Ательє надає послуги з пошиття та ремонту виробів. Основні сутності: клієнт, замовлення, виріб, послуга, матеріал, майстер, платіж. Замовлення має статуси: нове, у роботі, готове, видане, скасоване.

\subsection{Зацікавлені сторони}
\begin{itemize}
  \item Клієнт — оформлює замовлення, отримує виріб, здійснює оплату;
  \item Адміністратор — приймає замовлення, веде клієнтську базу, формує рахунки;
  \item Майстер — виконує роботи, фіксує етапи та витрати матеріалів;
  \item Власник — аналізує звітність, встановлює прайс-листи.
\end{itemize}

\subsection{Функціональні вимоги}
\begin{itemize}
  \item Реєстрація клієнтів та їх контактних даних;
  \item Оформлення замовлень з переліком робіт і матеріалів;
  \item Калькуляція вартості; облік платежів (передплата, остаточний розрахунок);
  \item Планування та відстеження етапів виконання робіт;
  \item Формування звітів (виручка, завантаженість майстрів, популярні послуги).
\end{itemize}

\subsection{Нефункціональні вимоги}
\begin{itemize}
  \item Зручність інтерфейсу для настільних браузерів;
  \item Збереження даних у реляційній БД; резервне копіювання;
  \item Аудит змін замовлень та платежів;
  \item Ролі та доступи (адміністратор, майстер, власник).
\end{itemize}


% 2. Проєктування системи (діаграми)
\section{Проєктування системи}
\subsection{Діаграма прецедентів}
На рис.~\ref{fig:usecase} подано узагальнену діаграму прецедентів.

\begin{figure}[H]
  \centering
  \begin{tikzpicture}[node distance=1.6cm]
    % Actors
    \node (client) [draw, rounded corners, align=center] {Актор\\Клієнт};
    \node (admin) [draw, rounded corners, align=center, below=of client] {Актор\\Адміністратор};
    \node (master) [draw, rounded corners, align=center, below=of admin] {Актор\\Майстер};
    % System boundary
    \node (sys) [draw, minimum width=9cm, minimum height=6cm, right=3.8cm of admin] {};
    \node at ($(sys.north)+(0,-0.4)$) {ІС Ательє};
    % Use cases
    \node (uc1) at ($(sys.west)!0.5!(sys.east)+(0,1.8)$) [draw, ellipse, align=center, minimum width=3.8cm] {Оформити\\замовлення};
    \node (uc2) [draw, ellipse, below=1.1cm of uc1, minimum width=3.8cm, align=center] {Розрахувати\\вартість};
    \node (uc3) [draw, ellipse, below=1.1cm of uc2, minimum width=3.8cm, align=center] {Виконати\\роботи};
    \node (uc4) [draw, ellipse, below=1.1cm of uc3, minimum width=3.8cm, align=center] {Облік\\платежів};
    % Associations
    \draw[-{Latex}] (client.east) -- (uc1.west);
    \draw[-{Latex}] (admin.east) -- (uc1.west);
    \draw[-{Latex}] (admin.east) |- (uc2.west);
    \draw[-{Latex}] (master.east) -- (uc3.west);
    \draw[-{Latex}] (admin.east) |- (uc4.west);
  \end{tikzpicture}
  \caption{Діаграма прецедентів для ІС ательє}
  \label{fig:usecase}
\end{figure}

\subsection{Діаграма активності: обробка замовлення}
\begin{figure}[H]
  \centering
  \begin{tikzpicture}[node distance=1.4cm]
    \tikzset{act/.style={draw,rounded corners,minimum width=3.6cm,align=center}};
    \node (start) [circle, fill=black, minimum size=6pt, inner sep=0pt] {};
    \node (a1) [act, below=of start] {Прийняти замовлення};
    \node (a2) [act, below=of a1] {Оцінити вартість \\ і терміни};
    \node (a3) [act, below=of a2] {Призначити майстра};
    \node (a4) [act, below=of a3] {Виконати роботи};
    \node (a5) [act, below=of a4] {Прийняти оплату \\ і видати виріб};
    \node (end) [circle, draw, minimum size=8pt, inner sep=0pt, below=of a5] {};
    \draw[-{Latex}] (start) -- (a1);
    \draw[-{Latex}] (a1) -- (a2);
    \draw[-{Latex}] (a2) -- (a3);
    \draw[-{Latex}] (a3) -- (a4);
    \draw[-{Latex}] (a4) -- (a5);
    \draw[-{Latex}] (a5) -- (end);
  \end{tikzpicture}
  \caption{Діаграма активності процесу замовлення}
  \label{fig:activity}
\end{figure}

\subsection{Діаграма класів}
\begin{figure}[H]
  \centering
  \begin{tikzpicture}[node distance=1.6cm]
    \tikzset{class/.style={draw,rectangle,minimum width=4cm,align=left}}
    \node (client) [class] {\textbf{Клієнт}\\\hrulefill\\+id: UUID\\+ПІБ: string\\+телефон: string};
    \node (order) [class, right=3.5cm of client] {\textbf{Замовлення}\\\hrulefill\\+id: UUID\\+дата: date\\+статус: enum\\+сума: money};
    \node (service) [class, below=of order] {\textbf{Послуга}\\\hrulefill\\+id: UUID\\+назва: string\\+ціна: money};
    \node (master) [class, below=of client] {\textbf{Майстер}\\\hrulefill\\+id: UUID\\+ПІБ: string\\+кваліфікація: string};

    % Relations
    \draw[-{Latex}] (client) -- node[above,sloped]{1..*} (order);
    \draw[-{Latex}] (order) -- node[right]{*} (service);
    \draw[-{Latex}] (master) -- node[above,sloped]{1..*} (order);
  \end{tikzpicture}
  \caption{Спрощена діаграма класів}
  \label{fig:class}
\end{figure}

\subsection{Діаграма послідовностей: оформлення замовлення}
\begin{figure}[H]
  \centering
  \begin{tikzpicture}[node distance=0.8cm]
    \tikzset{lifeline/.style={draw,minimum width=2.4cm,align=center}}
    \node (actor) [lifeline] {Клієнт};
    \node (ui) [lifeline, right=2.5cm of actor] {UI};
    \node (svc) [lifeline, right=2.5cm of ui] {Сервіс};
    \node (db) [lifeline, right=2.5cm of svc] {БД};

    % Lifelines
    \draw[dashed] (actor.south) -- ++(0,-5.5);
    \draw[dashed] (ui.south) -- ++(0,-5.5);
    \draw[dashed] (svc.south) -- ++(0,-5.5);
    \draw[dashed] (db.south) -- ++(0,-5.5);

    % Messages
    \draw[-{Latex}] (actor) -- node[above]{ввести дані} (ui);
    \draw[-{Latex}] (ui) -- node[above]{createOrder()} (svc);
    \draw[-{Latex}] (svc) -- node[above]{INSERT} (db);
    \draw[-{Latex}] (db) -- node[below]{OK} (svc);
    \draw[-{Latex}] (svc) -- node[below]{orderId} (ui);
    \draw[-{Latex}] (ui) -- node[below]{підтвердження} (actor);
  \end{tikzpicture}
  \caption{Діаграма послідовностей}
  \label{fig:sequence}
\end{figure}

\subsection{ER-діаграма бази даних}
\begin{figure}[H]
  \centering
  \begin{tikzpicture}[node distance=1.6cm]
    \tikzset{ent/.style={draw,rectangle,minimum width=3.8cm,align=center}};
    \tikzset{rel/.style={draw,diamond,aspect=2,align=center,inner sep=1pt}};

    \node (client) [ent] {CLIENT\\id, name, phone};
    \node (order) [ent, right=3.5cm of client] {ORDER\\id, date, status, sum, clientId};
    \node (service) [ent, below=of order] {SERVICE\\id, title, price};
    \node (orderitem) [ent, below=of client] {ORDER\_ITEM\\orderId, serviceId, qty, price};

    \node (r1) [rel, right=1.6cm of client] {places};
    \node (r2) [rel, below=0.8cm of r1] {contains};

    \draw (client) -- node[above]{1} (r1);
    \draw (r1) -- node[above]{N} (order);
    \draw (order) -- node[right]{1} (r2);
    \draw (r2) -- node[right]{N} (orderitem);
    \draw (service) -- node[right]{1} (r2);
  \end{tikzpicture}
  \caption{ER-діаграма БД}
  \label{fig:er}
\end{figure}

\subsection{Діаграма компонентів (структури)}
\begin{figure}[H]
  \centering
  \begin{tikzpicture}[node distance=1.6cm]
    \tikzset{comp/.style={draw,rectangle,minimum width=3.8cm,align=center}};
    \node (web) [comp] {Web UI};
    \node (api) [comp, right=3.5cm of web] {REST API};
    \node (db) [comp, right=3.5cm of api] {RDBMS};

    \draw[-{Latex}] (web) -- (api);
    \draw[-{Latex}] (api) -- (db);
  \end{tikzpicture}
  \caption{Діаграма компонентів}
  \label{fig:components}
\end{figure}


% 3. Реалізація (стисло)
\section{Реалізація}
Програмна реалізація може бути виконана у вигляді веб-додатка з архітектурою клієнт–сервер. Клієнтська частина (наприклад, React) взаємодіє з серверним REST API (наприклад, Node.js/Express, Java Spring або Python FastAPI), дані зберігаються у реляційній СУБД (PostgreSQL/MySQL). Аутентифікація може базуватися на JWT, авторизація — на ролях.


% 4. Тестування (стисло)
\section{Тестування}
Запропоновано рівні тестування: модульне (сервісні методи калькуляції, валідації), інтеграційне (взаємодія API та БД), системне (сценарії оформлення замовлення, оплати), приймальне (з користувачами). Для не функціональних аспектів — навантажувальне тестування ключових ендпойнтів.


% Висновки
\section{Висновки}
Було проаналізовано предметну область ательє, уточнено вимоги та розроблено модель системи за допомогою UML і ER-діаграм. Запропоновано архітектуру рішення та підходи до тестування. Результати можуть бути використані як основа для подальшої детальної реалізації.


% Список використаних джерел
\printbibliography[heading=bibintoc,title={Список використаних джерел}]

% Додатки (за потреби)
% \appendix
% \section{Додаток А}

\end{document}
